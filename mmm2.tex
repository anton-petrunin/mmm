
%amslatex file
%\documentclass[amssymb,12pt,amscd,leqno]{amsart}
\documentclass[12pt,leqno]{amsart}
\usepackage[dvips]{graphics}
\usepackage{amssymb}
\usepackage{hyperref}
\usepackage{color}
%\newenvironment{beweis}{\begin{proof}[Beweis]}{\end{proof}}
%\usepackage{epsfig}
%\typein[\answer]{To run syntax check, enter `S';
% otherwise, just press return:}
% \let\SANSWER\syntaxonly     \let\sANSWER\syntaxonly
% \csname\answer ANSWER\endcsname
\newenvironment{pf}{\proof[\proofname]}{\endproof}

\numberwithin{equation}{section}

%\voffset=-.7in
%\hoffset=-.7in
%\setlength{\textwidth}{7in}

%\setlength{\textheight}{8.5in}

\newtheorem{thm}{THEOREM}[section]
\newtheorem{thmmain}[thm]{\tref{thm-main}}%[section]
\renewcommand{\thethmmain}{}
\newtheorem{lem}[thm]{Lemma}
%\newtheorem{defn}[thm]{DEFINITION}
\newtheorem{cor}[thm]{Corollary}
\newtheorem{prop}[thm]{PROPOSITION}
\newtheorem{conj}[thm]{CONJECTURE}
\newtheorem{quest}[thm]{PROBLEM}
\newtheorem{fak}[thm]{Fakt}

\newtheorem{hyp}[thm]{HYPOTHESES}
\newtheorem{ft}[thm]{FAKT}
\newtheorem{satz}[thm]{SATZ}
 \theoremstyle{definition}
\newtheorem{defn}[thm]{Definition}%[section]
%\newtheorem*{proof}{Beweis}
\theoremstyle{remark}
\newtheorem{rem}[thm]{Remark}
\newtheorem{ex}[thm]{Example}

\newcommand{\tref}[1]{Theorem~\ref{#1}}
\newcommand{\cref}[1]{Corollary~\ref{#1}}
\newcommand{\pref}[1]{Proposition~\ref{#1}}
\newcommand{\rref}[1]{Remark~\ref{#1}}
\newcommand{\dref}[1]{Definition~\ref{#1}}
\newcommand{\secref}[1]{\S\ref{#1}}
\newcommand{\lref}[1]{Lemma~\ref{#1}}
\newcommand{\exref}[1]{Example~\ref{#1}}
\newcommand{\sref}[1]{Satz~\ref{#1}}
\newcommand{\fref}[1]{Fakt~\ref{#1}}
\newcommand{\Ant}{\mathrm{Ant}}
\newcommand{\diam}{\mathrm{diam}}
\newcommand{\Pol}{\mathrm{Pol}}
\newcommand{\vol}{\mathrm{vol}}
\newcommand{\pt}{\mathrm{pt}}
\newcommand{\rad}{\mathrm{rad}}
\newcommand{\scal}{\mathrm{scal}}
\newcommand{\Rad}{\mathrm{Rad}}
\newcommand{\Sub}{\mathrm{Sub}}
\newcommand{\Iso}{\mathrm{Iso}}
\newcommand{\Rg}{\mathrm{Rg}}
\newcommand{\Fix}{\mathrm{Fix}}
\newcommand{\R}{\mathbb{R}}

\def\X{{X^\varepsilon}}
\def\di{{\bold div}\,}

\def\RR{\mathbb R}
\def\co{\colon\thinspace}
\def\eps{\varepsilon}
\def\BAD{\mathop{\rm BAD}\nolimits}%


\def\:{\colon}
\newcommand*{\set}[2]{\left\{\,\left.{#1}\vphantom{#2}\,\right|\,{#2}\,\right\}}
\def\emptyset{\varnothing}
\newcommand*{\GHto}{\mathbin{\begin{picture}(16,3.5)
\put(-1,1){$\longrightarrow$}
\put(13,-3){\llap{\text{\sf\tiny GH}}}
\end{picture}}}


\newcommand{\curv}{\mathop{\rm curv}}
%MY STYLE
\def\parit#1{\medskip\noindent{\it #1}}
\def\parbf#1{\medskip\noindent{\bf #1}}
\def\qeds{\qed\par\medskip}
\def\qedsf{\vskip-6mm\qeds}


\pagestyle{plain}

\begin{document}
%\tableofcontents
\pagebreak
%\bibliographystyle{alpha}

%\pagenumbering{roman}

\title{Volume of balls and geodesic flow  on Alexandrov spaces}

\author{Vitya Kapovitch, Alexander Lytchak and Anton Petrunin}
%\address{Mathematisches Institut\\ Universit\"at Bonn\\
%Wegelerstrasse 10, 53115 Bonn, Germany\\}
%\email{lytchak\@@math.uni-bonn.de}


%\subjclass{53C20,  52B99}
%\footnotetext[1]{}


%\keywords{Semi-convex functions, Alexandrov spaces, differentials}
%Spherical building, Euclidean building, non-positive curvature}

%\thanks{to Professor Ballmann}

%\date{\today}
%\date{August, 1997}

\begin{abstract}
 We relate the existence of many infinite geodesics on Alexandrov spaces to a statement about the average growth of balls. We deduce that the geodesic flow exits and preserves the Liouville measure in several important cases. The developed analytic tool has close ties  to integral geometry.
\end{abstract}


\maketitle
\renewcommand{\theequation}{\arabic{section}.\arabic{equation}}
\pagenumbering{arabic}

%\tableofcontents

\section{Introduction}
\subsection{Motivation and application}
The following structural questions, which  have been formulated early on in the development of   the theory of Alexandrov spaces (see \cite{BGP} and \cite{P2}), are seemingly out of reach by purely geometric methods:
\begin{itemize}
\item Is the boundary of an Alexandrov space an Alexandrov space?
\item Are there  ``many'' infinite geodesics on any Alexandrov space without boundary?
\end{itemize}

In this paper we address the latter question, and obtain an affirmative answer in several cases.
We also develop an analytic tool which has a formal similarity with  classical integral geometry
and might  be interesting in its own right beyond the realm of Alexandrov geometry.

In particular, we prove the existence of such  infinite geodesics in the most classical examples of non-smooth Alexandrov spaces:

\begin{thm} \label{thmfirst}
Let $X$ be the boundary of a convex body in $\mathbb R^{n+1}$.
Then almost any direction in the tangent bundle $TX$ of $X$
is the starting direction of a unique  infinite geodesic on $X$.
The geodesic flow defined on a subset  of full  measure in $TX$ preserves the Liouville measure.
\end{thm}

Apparently, even the existence of a single infinite geodesic has not been known,  even in the case $n=2$.


 \subsection{The geodesic flow and mm-boundary}
In a smooth manifold with boundary the geodesic flow is not defined for all times and the amount of geodesics terminating at the boundary in a given time interval depends on the size  of this boundary.
In the following definition we grasp the size of the boundary by estimating the average volumes of small balls and their deviations from the corresponding  volumes in the Euclidean space.

Let $(X,d)$ be a locally compact separable metric space. Let $\mu$ be   a  Radon  measure on $X$ and assume that closed bounded subsets
of $X$ have finite $\mu$-measure.
For $x\in X$ and $r>0$
denote by $B_r (x)$ the open metric ball of radius $r$ around  $x$. Define the lower semicontinuous functions $b_r:X\to [0,\infty )$ by
\begin{equation}
 b_r(x):=\mu (B_r (x)) \; .
 \end{equation}
For a fixed integer $n$ (the ``dimension'' of $X$)
 we define a signed measure $v_r$ on $X$, absolutely continuous with respect to $\mu$,   as
\begin{equation} \label{eq:first}
 v_r=v^n_r = (1 - \frac {b_r} { \omega _n r^n} )\cdot \mu \, ,
\end{equation}
where $\omega _n$ is the volume of the $n$-dimensional unit Euclidean ball.

A signed measure, here and below, is defined  as an element of the space $\mathcal M(X)$ dual to the space of compactly supported continuous functions $C_c (X)$.
We always consider $\mathcal M(X)$ with the topology of weak convergence.

In all examples and applications below,  $n$  is the Hausdorff dimension of $X$ and $\mu$ is the $n$-dimensional Hausdorff measure $\mathcal H^n =\mathcal H^n _X$ on $X$, always normalized such that it coincides with the Lebesgue measure on $\R^n$.


The map $\mathbb{R}_+\to\mathcal M(X)$ defined as $r\mapsto  v_r=v_r ^n$ controls in a very rough integral sense the deviation of $X$ from $\R^n$.
The Taylor coefficients of $v_r$ at $0$ (if they are defined) can be interpreted as integral curvatures, in analogy with integral geometry. The following  example is well known.



    \begin{ex} \label{smoothscal}
 Let $X^n$ be a  smooth Riemannian manifold with  Riemannian volume $\mu =\mathcal H^n$.  Then, for $r\to 0$,
  the measures $v_r /r^2$ converge in $\mathcal M(X)$ to $\frac 1 {6(n+2)}\cdot scal \cdot \mu$, where
  $scal$ denotes the scalar curvature of $X$.
\end{ex}



The next less known example can be obtained by computations in local coordinates. Since it is not needed in the sequel, we omit
  the details.




  \begin{ex} \label{mainex}
Let $X$ be a smooth $n$-dimensional Riemannian manifold with boundary $\partial X$.
Then, for $r\to 0$, the measures   $v_r/r$  converge in $\mathcal M(X)$ to
$c_n \cdot \mathcal H^{n-1} _{\partial X}$, for some constant $c_n >0$ depending only on $n$.
\end{ex}

%Here and below we consider the space of signed measures $\mathcal M (X)$ with the topology of weak convergence.
This example  suggests  to view the first Taylor coefficient of $v_r$   as the ``boundary'' of the metric-measure space $(X,\mu,d)$,
 which we make more precise by the following central definition of this paper.
% In the next definition, we say that a family of measures on a locally compact space $X$
%  is locally bounded if for any compact subspace $K$ of $X$, the restrictions of the family to $K$ are bounded.

\begin{defn}  \footnote{A.:  Where should one introduce ``total variations''?  Should one define the mm-boundary always as limsup?{\color{red} I think it's ok as is. Vitya}}
Let $(X,d,\mu)$ be a metric measure space as above,  let  the signed Radon measures $v_r$ be as in \eqref{eq:first} and let  the Radon measure $|v_r|$ be the total variation of $v_r$.
If the  family of Radon measures
\[\{\, \frac {| v_r |}  {r } \,;\,  0<r\leq  1 \,\}\]
has an upper bound in $\mathcal M( X)$ we say that $X$ has   locally finite mm-boundary.
If the signed measures $v_r  /r$  converge for $r\to 0$ to a measure $\nu$, we call $\nu$ the mm-boundary of $X$.  If $\nu =0$ we say that $X$ has trivial mm-boundary.
\end{defn}


We refer to Subsection \ref{subsec:example}  and Section \ref{sec:final}  below for a discussion of  examples and questions, and
 state now our central  result connecting  mm-boundaries to the existence  of infinite geodesics in Alexandrov spaces:

 \begin{thm} \label{thmmain}
 Let $X$ be an Alexandrov space. If $X$ has trivial mm-boundary, then  almost each direction of the tangent
 bundle $TX$ is the starting direction of  an infinite local geodesic. In this case, the geodesic flow preserves the Liouville measure on $TX$.
 \end{thm}

\subsection{Size of the mm-boundary in Alexandrov spaces}
The next theorem shows that, similarly to \exref{mainex},  the topological boundary is closely related to the mm-boundary in Alexandrov spaces.

\begin{thm} \label{alexandrovthm}
Let $X^n$ be an Alexandrov space. Then $X$ has locally finite mm-boundary.  \footnote{ A.:  Is it better to formulate for limsup or limits of subsequences?}
If $\nu$ denotes any limit of a subsequence $\lim _{s\to 0} \frac {v_s}  {s}$ then $\nu$ is a Radon measure and
 the following holds true.
\begin{enumerate}
\item\label{full-measure-zero-nu}There is a Borel set $A_0$ of full $\mathcal H^n$ measure in $X$
with $\nu (A_0)=0$.
\item\label{bry-nu} If the topological boundary $\partial X$ is non-empty then $\nu \geq c \cdot \mathcal H^{n-1} _{\partial X}$,
for a positive constant $c$ depending only on $n$.
\item \label{n-1-nu} If the topological boundary $\partial X$ is empty then $\nu (A)=0$, for any Borel subset $A\subset X$ with  $\mathcal H^{n-1} (A)<\infty$.
\end{enumerate}
\end{thm}

 We strongly believe that an Alexandrov space with empty topological boundary $\partial X$  has a trivial mm-boundary,
which would solve the question about the existence of infinite geodesics.
We could prove this conjecture in two special cases:



\begin{thm} \label{hypersurface}
Let $X^n$ be a convex hypersurface in $\R^{n+1}$  or let $X$ be a two-dimensional  Alexandrov space without boundary. Then $X$ has trivial mm-boundary.
\end{thm}


In combination with \tref{thmmain} this proves \tref{thmfirst}.
The two-dimensional case
will follow from a much stronger result discussed in the next subsection.
Another proof can be built on an analog of \tref{thmfirst} for the hyperbolic space,
since by  Alexandrov's embedding theorem, any two-dimensional Alexandrov space without boundary
is locally isometric to a convex surface in $\mathbb H^3$.



\subsection{mm-curvature}
Motivated by Example \ref{smoothscal} one can naively hope that the second Taylor coefficient at $0$ of the
map $v_r\colon r\mapsto \mathcal M(X)$ describes the scalar curvature of the space.
%This again has  natural analogy with the classical integral geometry.



\begin{defn}
 Let $X,v_r$ be as above. If the family $v_r /r^2, r\leq 1$ is locally bounded we say that $X$ has locally finite  scalar curvature.
 % For any compact subspace $K$ of $X$, we call $\limsup _{r\to 0} \|v_r /r\| _K$ the bound of the scalar curvature  in $K$.
  If the measures $v_r /r^2$ locally converge to a measure $\nu$, we call $\nu$ the mm-curvature of $X$.
\end{defn}


Clearly, local finiteness of mm-curvature as defined above implies that the mm-boundary is trivial.
Thus, the following result proves \tref{hypersurface} in the $2$-dimensional case.


\begin{thm} \label{intsurface}
 Let $X$ be a 2-dimensional Alexandrov space without boundary.
 Then $X$ has locally finite mm-curvature.
% Moreover, it is bounded by a linear function of $|\omega |$, at least for
% if $|\omega|$ is small enough.
\end{thm}


This finiteness result holds true in the much greater generality of surfaces  with bounded integral curvature in the sense of Alexandrov--Zallgaler--Reshetnyak \cite{Reshetnyak-GeomIV}, \cite{AZ}, see Section \ref{sec:surface} below.

Note, however, that the mm-curvature
in \tref{intsurface} does not need to coincide with the ``curvature measure'' as it defined in \cite{Reshetnyak-GeomIV},   even in the case of a cone, cf. Example \ref{ex:cone}.  In particular, this shows that the mm-curvatures in $2$-dimensional Alexandrov spaces are not stable under Gromov--Hausdroff convergence.


\begin{rem}
N. Lebedeva and the third named author have found in \cite{LP}  a ``scalar curvature measure'' on all smoothable Alexandrov spaces, which has the property of being stable
under non-collapsed Hausdroff convergence. There is a hope, justified by our proof of  \tref{intsurface}, that a better understanding of this ``stable curvature measure'' will
lead to some control of the geometric notions of mm-boundary and mm-curvature discussed here.
\end{rem}



\subsection{Analogy with the Killing--Lipschitz curvatures and the proof of \tref{thmmain}}
For a compact convex body or compact smooth submanifold $M$ in $\R^n$ one classically considers
the  volumes $w(r)=\mathcal H^n (B_r(M))$ of distance tubes $B_r(M)$ around $M$.
It turns out that $w(r)$ is a polynomial, at least for small $r$, and  the coefficients of
$w$, called the   Killing--Lipschitz curvatures of $M$, are given by  integrals over some intrinsically defined curvature
 terms.  Moreover, these coefficients can be localized and considered as measures on $M$.

To make the formal similarity with our approach  to mm-boundary and mm-curvature more transparent, we observe that
(at least for a smooth $n$-dimensional manifold $M$) the number   $\int _M \mu (B_r(x)) d\mu (x)$  can be interpreted as the $\mathcal H^{2n}$-measures
of the distance tubes $B_{r/\sqrt 2} (\Delta )$ around the diagonal $\Delta $ in the Cartesian product $M\times M$.

The natural interpretation of the tangent bundle of $M$ as the normal  bundle of the diagonal $\Delta$ in $M\times M$ serves us as the connection between
the measure theoretical properties of the tubes around $M$ and the dynamical properties of the geodesic flow.
We clarify this abstract statements by explaining the main idea of  our  proof of \tref{thmmain} in the (trivial and well-known) case of a smooth Riemannian manifold $X=M$.
Thus, we just sketch a new proof,  of the fact   that the geodesic flow $\phi$ of a smooth Riemannian manifold $M$ preserves the Liouville measure $\mathcal M$ on $TM$.
This proof is stable and rough enough to be transferred to the singular situation,
% but
%The idea can be fastest explained  in the case where $X$ is a smooth Riemannian manifold $M$,
%where it provides a metric-measure  proof of the most classical  statement that the geodesic flow $\phi_t\:TM\to TM$ preserves the Liouville  measure $\mathcal M$ on $TM$. It suffices to consider the case $t=1$
%and to prove the statement on arbitrary small neighborhoods of the $0$-section.
%For any natural $n>0$, we can write $\phi _t =\phi _{\frac t n} ^n$, thus it is sufficient  to understand this measure preservance for small $t$, or even infinitesimally.

Let  $\phi_t\: TM\to TM$ be the geodesic flow for time $t$.
Denote by $\pi\: TM\to M$ the tangent bundle of $M$. Define the map $E\:TM\to M\times M$  by
\[E(v)=(\pi(v), \pi (\phi _1 (v)).\]
By construction, $E(-\phi_1(v))= J(E(v))$,
where $J$ is the involution of $M\times M$ which switches the coordinates.
Since $J$ preserves $\mathcal H^{2n}$ on $M\times M$ and $v\to -v$ preserves the Liouville measure $\mathcal M$ on $TM$,
the statement that $\phi$ is measure preserving hinges upon the smallness of measure-distortion of the map $E \:(TM,\mathcal M) \to (M\times M, \mathcal H^{2n})$ close to the $0$-section.
In the present case of Riemannian manifold,
this property of $\phi_1$ is expressed by  the fact that the differential of $E$ is the identity (after suitable identifications).
In the general case of Alexandrov spaces, we similarly observe that the ``infinitesimal'' deviation (via the canonical map $E$) between $J$ being measure preserving (which we know) and  $\phi _1$ being measure preserving (which is what we want to show)
is expressed precisely as the triviality of  the mm-boundary.






  %In fact, we  restrict here  to the case of a smooth Riemannian manifold $M$ and provide a metric explanation for the invariance of the Liouville measure under the geodesic flow $\phi _t$.  Note that $   We discretize  of the geodesic flow and think of it as the limiting process, for $t\to 0$ of going
%in the classical case of a smooth Riemannian manifold $M$.






\subsection{Relations with quasi-geodesics}
In Alexandrov spaces  many geodesics cannot be extended as local geodesics beyond some point.
This sudden ``death''  of geodesics  causes  serious technical difficulties.
To deal with this problem  so called \emph{quasi-geodesics} (certain generalized geodesic)
were introduced
and it was proved
that any direction is the starting direction of an infinite quasi-geodesic;
see \cite{Petsemi}, \cite{PP} and the references therein.

For most application the existence of quasi-geodesics is sufficient.
The quasi-geodesics are in many senses as good as geodesics, and even better in the sense of their stability.
However, the quasi-geodesic in the given direction are not uniquely defined, they can branch.
One motivation for the present paper was an attempt to prove Liouville's theorem for the ``quasi-geodesic flow''.

Unlike local geodesics, quasi-geodesics are defined by a global metric condition. Therefore they enjoy much better stability properties
and can often be found by limiting arguments.
We note that while many Alexandrov spaces appear naturally  as limits of smooth Riemannian manifolds, the properties of the geodesic flow, mm-boundaries and mm-curvature are unstable under
convergence.
For instance, there is  no chance to deduce our main results about convex hypersurfaces by approximating them by smooth or polyhedral ones.



\subsection{Examples} \label{subsec:example}
The estimates of the   mm-boundary and mm-curvature are quite involved even in  quite  simple situations.
We collect a few instructive examples. %indicating relations with curvature measures in integral geometry.
The examples  are not  needed in the sequel and we omit the somewhat tedious computations. \footnote{A.: How instructive are the examples? All of them?{\color{red} I think the  first one can be dropped. The last 3 I would keep. Not sure about keeping the second one without a proof. Vitya}
 A.:   But one also might add the following one, together with the comment, that $\mu =\mathcal H^n$ is the only measure for which our considerations are of some interest.  Maybe, it is better not to say it.  \begin{ex}
Let $X$ be a Lipschitz manifold.
Then $\lim _{r\to 0} v_r =0$.
\end{ex}
}











\begin{ex}
Let $X$ be a Riemannian manifold with a Lipschitz continuous metric.
Then $X$ has  trivial mm-boundary.
\end{ex}


 \begin{ex}
  If $X$ is a manifold with  both-sided bounded curvature in the sense of Alexandrov then its
  mm-curvature   is  well-defined and  absolutely continuous with respect to the Hausdorff measure.
 \end{ex}




\begin{ex} \label{ex:cone}
 Let $X$ be the flat cone $CS_{\rho}$ over a circle of length $\rho$.
  Then $X$ has finite mm-curvature. The mm-curvature is a Dirac measure of mass $m(\rho)$ concentrated in the origin.
  If $\alpha =2\pi-\rho$ denotes the defect of $X$ then the classical curvature measure is the Dirac measure of total mass
  $\frac \alpha 6$.  On the other hand $m(\rho)= \frac \alpha 6 + f(\alpha)$, where
  $f(\alpha)$ is a non-zero function which has at the origin the form $f(\alpha)= O(\alpha ^3)$.
   \end{ex}




 \begin{ex}
 Let $X$ be a finite $n$-dimensional simplicial complex with an intrinsic metric $d$.
 Assume that the restriction of $d$ to each simplex is given by a (sufficiently) smooth Riemannian metric.
 Then $X$ has a finite boundary $\nu$, concentrated on  the $(n-1)$-skeleton $X^{n-1}$.
 Moreover, the restriction of $\nu $ to
 $X^{n-1}$ is given by $\nu= c_n \cdot f \cdot \mathcal H ^{n-1}$.
 Here $c_n$ is the positive constant from Example  \ref{mainex} and  $f$ is the locally constant function $f(p)=(2-m)\cdot m$, where $m$ is the number of top-dimensional simplices
 adjacent to a point $p\in X^{n-1}$.
 \end{ex}


 \begin{ex}
 Assume that $X$ is as above  and that it  is a pseudo-manifold, hence has trivial mm-boundary.
Then $X$ has finite mm-curvature $\nu$. \footnote{A.: I have killed the long description of the mm-curvature?  Would it be better to keep it? {\color{red} Either way is fine with me. Vitya}}
% given by $\mu = \mu _1 + \mu _2 + \mu _3$.  Here $\mu_1$ is  the usual scalar
%    curvature on maximal faces. The second term  $\mu _2$ lives on $ X^{n-1}$, and is given by
%    $q_n \cdot m(p) \cdot \mathcal H^{n-1}$.
%    Here $m(p)$ is the sum of mean curvatures and $q_n$ is a constant depending only on the dimension.
%    Finally, $\mu_3$ lives on the $(n-2)$-skeletton  $X^{n-2}$ and is given by $w_n \cdot c(p) \cdot \mathcal H^{n-2}$.
%     Here $w_n$ is a constant depending on the dimension and $c(p)$ depends only on the link at $p$.
 \end{ex}



\subsection{Structure of the paper} \footnote{New subsection. Is it useful?}
After preliminaries collected in Section \ref{sec:prelim}, we prove \tref{thmmain} in Section \ref{sec:Liou} along the lines sketched above.
In Sections \ref{sec:surface}, \ref{sec:hyper}  and \ref{sec:Alex} we prove  the remaining Theorems \ref{intsurface}, \ref{hypersurface}, \ref{alexandrovthm}, respectively. The proofs of these theorems all rely on a decomposition of the space into a regular and singular part, with
a quantitative estimate of the size of the singular part. Finally, on the regular part we estimate the mm-curvature and mm-boundary through other
canonical measures  on these spaces. In the case of surfaces, this comparison measure is the classical curvature measure,  in the case of convex hypersurfaces, this comparison measure is the mean curvature.  Finally, in the case of a general  Alexandrov space, the comparison is given by the derivative of the metric tensor, when the last one is given in canonical DC-coordinates defined by Perelman. The needed control of the ball growth
in terms of these measures is given by a theorem of M. Bonk  and U. Lang in the case of surfaces  and follows from classical convex geometry in the case of hypersurfaces. The analytical comparison result  needed for Alexandrov spaces is established in Section \ref{sec-BV-estimate}. In the final
Section \ref{sec:final} we collect some comments and questions which naturally arose during the work on this paper. \footnote{Final sentence is a bit stupid}




\subsection{Acknowledgments}  The authors are grateful for helpful conversations and comments to Andreas Bernig, Koichi Nagano,  .....
The first author was supported in part by grants from NSERC and the Simons foundation. The second author was supported in part by the DFG grants  SFB TRR 191 and SPP 2026. \footnote{A.: All aknowledgements here? Or separate between people and institutions?}


\section{Preliminaries} \label{sec:prelim}
\subsection{Metric spaces}
We refer to \cite{BBI01} for basics on metric spaces and to \cite{Federer} for basics on measure theory needed here.
The distance between points $x,y$ in a metric space $X$ will be denoted by $d(x,y)$ or $|xy|_X$ or just $|xy|$.
By $B_r(x)$ we will denote the open metric ball of radius $r$ around a point $x$. For $A\subset X$
we denote by $B_r (A)$ the open tubular neighborhood $B_r (A) =\cup _{x\in A} B_r (x)$ around $A$.


A geodesic $\gamma$ in a metric space $X$ is  a map $\gamma \:I\to X$ defined on an interval $I \subset \R$ such that for some number $\lambda \geq 0$ one has
$d(\gamma (t),\gamma (s)) =\lambda \cdot |t-s|$ for all $t,s\in I$. In particular, we allow $\gamma$ to have any constant velocity $\lambda \geq 0$. On the other hand, geodesics
as defined here will always be globally ``minimizing geodesics'' in the usual notations of Riemannian geometry.
A local geodesic is a curve $\gamma\: I\to X$ such that its restriction to a small neighborhood of any point in $I$ is  a geodesic. Note that a local geodesic is a curve of constant velocity.

\subsection{Metric measure spaces}
Let $X$ be a locally compact separable metric space. A Radon measure on $X$ is a measure on $X$ for which all compact subsets are measurable and have finite measure.
Any Radon measure is canonically contained in the dual space $\mathcal M(X)$ of the topological vector space $\mathcal C_c (X)$ of compactly supported continuous functions on $X$.
All elements in $\mathcal M(X)$ are called signed Radon measures. For any signed Radon measure $\mu$ there exist unique Radon measures $\mu ^{\pm}$ such that $\mu =\mu ^+- \mu ^-$.
In such a situation, we will denote by $|\mu |$ the Radon measure $\mu ^+ +\mu ^-$ and call it the total variation of $\mu$.


%For a Radon measure $\mu$ on $X$, the $n$-dimensional density of $\mu$ at a point $x$ is defined by  $\limsup _{r\to 0} \frac {\mu (B_r(x))} {\omega _n r^n}$.

 A family   $\mathcal F$ of Radon measures on $X$ is bounded if there exists some Radon measure $\mu$ on $X$ such that $\mu \geq \nu $ for any $\nu \in \mathcal F$.  Let $\nu _s, 1\geq s >0$ be a bounded family of Radon measures. For any sequence $s_j\to 0$, the sequence $\nu _{s_j}$ has a convergent
subsequence. The set of all limits of all such subsequences is a bounded family of Radon measures and has a well defined least upper bound Radon measure $\mu$. In  this situation we will write $\mu =\limsup _{s\to 0} \nu _s$.

We will repeatedly use the following simple


\begin{lem} \label{lem:exchange}
Let $X$ be a metric space with two Radon measures $\mu $ and $\nu$. Let $r>0$ be arbitrary and let  $A\subset X$ be a Borel subset. Then
 $$\int _A \mu (B_r(x)) \, d\nu (x) \leq \int _{B_r (A)}  \nu (B_r (x)) \, d\mu (x) \; .$$
\end{lem}

\begin{proof} We apply Fubini's theorem twice.
On the right hand side is the volume of  $S=\{(y,x) \in X\times X| y \in A, d(y,x) <r \}$ with respect to the product measure $\nu \otimes \mu$.
On the left hand side is the volume of the larger set $T=\{(y,x) \in  X\times X| x\in B_r(A), d(y,x)<r \}$ with respect to the same measure.
\end{proof}

\subsection{Alexandrov spaces} \label{subsec:Alex}
We are assuming that the reader is familiar with basic theory of Alexandrov spaces. We refer to ~\cite{BGP} as a good introduction to the subject.
Throughout this paper, an Alexandrov space will denote a complete, locally compact, geodesic metric space of finite Hausdorff dimension and of curvature bounded from below by some
number $\kappa \in \R$.
%All Alexandrov spaces considered in this paper are finite dimensional.
When talking about Alexandrov spaces an upper index will indicate the Hausdorff dimension, that is, $X^n$ will denote an $n$-dimensional Alexandrov space, always equipped with the $n$-dimensional Hausdorff measure $\mathcal H^n$. In this situation we will sometimes denote
 $\mathcal H^n$ by $\vol_n$.

The set of starting directions of geodesics starting in a given point $x\in X$ carries a natural metric, whose completion is the tangent space $T_x=T_xX$
of $X$ at the point $x$.  It is again an $n$-dimensional Alexandrov space of non-negative curvature. Moreover, it is the Euclidean cone
over the space $\Sigma _x$ of unit directions. The Euclidean cone structure defines multiplications by positive scalars $\lambda \geq 0$ on $T_xX$. The origin
of the cone $T_xX$ is denoted by $0=0_x$. Elements of $T_xX$ are called tangent vectors at $x$. For $v\in T_xX$ the norm $|v|$ of $v$ is the distance of $v$ from the origin $0_x$.




Any two geodesics in $X$ with initial starting vectors coincide. Hence,
for any $v\in T_xX$ there exists at most one geodesic $\gamma _v$  starting in $x$ in the direction of $v$ and defined on a maximal possible interval $I =[0,t]$ with $t\in [0, \infty )$
or $I=[0,\infty)$.  Let $D_x$ denote the set of all vectors $v\in T_xX$ for which $\gamma _v(1)$ is defined.
We set $\gamma _v (1)=\exp_x (v)$. The map $\exp_x \:D_x \cap B_r(0_x) \to B_r(x)$ is surjective for any $r>0$.
 Moreover, for a constant $C\geq 0$ depending on $\kappa$ and all $r< \frac 1 C$ the map $\exp_x \:D_x \cap B_r(0) \to B_r(x)$ is  $(1+ C \cdot  r^2)$-Lipschitz continuous.



By the theorem of Bishop-Gromov, the volume $b_r (x)=\mathcal H^n (B_r(x))$ is bounded from above by the corresponding volume in the
space of constant curvature $\kappa$. In particular, $b_r (x) \leq \omega _n \cdot r^n + C \cdot r^{n+2}$ for all $r\leq \frac 1 C$, where
the constant $C$ can be chosen as before.  Thus, using the notations from the introduction, the signed measures $v_r  =v_r^n $ satisfy
$v_r \geq -Cr^2 \mathcal H^n$  for all sufficiently small $r$.
 Here and in the previous paragraph,  one can set $C=0$ if $\kappa \geq 0$.

 {\color{red} From now on we will assume that $k=0$ and $C=0$ in all proofs. This implies in particular that $v_r\ge 0$ and $\exp_x$ is $1$-Lipschitz. The proofs can be  easily adjusted to the case of general $k$ using the above estimate.}\footnote{{\color{red}I suggest we do all the proofs for $k=0$. Vitya}  It is fine with me. However, in the first 5 chapters it appears only here and  between euqation 3.4 and 3.5 in the proof of Theorem 3.5} 


Denote by $X_{reg}$ the set of all points $x\in X$ with $T_xX$   isometric to $\R^n$.  The set $X_{reg}$ has full $\mathcal H^n$-measure in $X$.
Any inner point of any geodesic starting on $X_{reg}$ is contained in $X_{reg}$, \cite{Petparallel}.


The topological boundary $\partial X$ of $X$ can be defined as the closure of the  set of all points $x\in X$ with $T_xX$ isometric to a Euclidean half-space. Up to a subset of Hausdorff dimension $n-2$, $\partial X$ is an $(n-1)$-dimensional Lipschitz manifold. On the other hand, $X\setminus (X_{reg} \cup \partial X)$ has Hausdorff dimension at most $n-2$.


\subsection{Volume and Bi-lipschitz maps}
It is easy to get an estimate of $v_r$  on parts which are Lipschitz manifolds and where one can control the biLipschitz constants.
Let $\mu =\mathcal H^n$ be a Radon measure on the metric space $X$.
Let $U\subset X$ and $V\subset \R^n$ be open and assume that there is an $(1+\delta)$-biLipschitz map
$f\:U \to V$.  %Assume $\delta  \leq \delta _n$, where $\delta_n$ is a sufficiently small constant depending on $n$.
 Let $A\subset U$ be given with
$B_{(1+\delta) r} (A) \subset U$ and $B_{(1+\delta) r} (f(A)) \subset V$.  Then, for all $x\in A$,
\begin{equation} \label{eq:bilip}
(1+\delta)  ^{-2n} \leq \frac {b_r(x)} {\omega _n r^n} \leq (1+\delta ) ^{2n} \; .
\end{equation}
Therefore, if $\delta $ is sufficiently small,  $\|v_r \| (A) \leq  4 n \cdot \omega _n \cdot  \delta \cdot \mathcal H^n (A).$
\begin{rem}  \footnote{A.: Tosha, do you know it is true?} For Alexandrov  spaces the opposite seem to hold (but I am not quite sure):
If the volume of  a ball $B_r (x)$ satisfies $\omega _n r^n -\mathcal H^n (B_r (x)) \leq \delta r ^n$, then
the ball $B_{\frac r 4} (x)$ is  $(1+C_n \delta )$-biLipschitz  to a Euclidean ball.
\end{rem}





\section{Liouville measure and geodesics} \label{sec:Liou}
\subsection{Tangent bundle and Liouville measure} \label{subsec:tb}
Let $X^n$ be an $n$-dimensional Alexandrov space. Consider the tangent bundle of $X$
  as the  disjoint union $TX=\cup T_x X$ of the tangent spaces at different points. Denote by $\pi\:TX\to X$ the canonical footpoint projection, $\pi (T_xX) =\{x\}$.
	For any subset $K\subset X$ denote by $TK$ the preimage $\pi^{-1} (K)= \cup _{x\in K} T_xX$. For $r>0$, denote by $T^r K$ the set of all vectors in $TK$
	of norm smaller than $r$.
% Let $X_{reg}\subset X$ be the set of all regular points and $TX_{reg} =\pi^{-1} (X_{reg}) \subset TX$.
 The   differentiable and Riemannian structure on the set of regular points discussed in \cite{Otsu}, cf. Section \ref{sec:Alex} below,  provide  $TX_{reg}$ with a canonical structure of a Euclidean vector bundle over $X_{reg}$. In this topology, for any sequence of geodesics $\gamma _i$ in $X_{reg}$ converging to a geodesic $\gamma$ the starting directions of $\gamma _i$ converge to the starting direction of $\gamma$.

 On the Euclidean vector bundle $TX_{reg}$ over $X_{reg}$ we have a canonical measure, which locally coincides with the product measure of $\mathcal H^n _X$ and the Lebesgue measures on the fibers. More precisely,
 it is the unique Borel measure $\mathcal M$ on $TX_{reg}$ such that for any  Borel set $A\subset TX_{reg}$
 $$\mathcal M(A)= \int _X (\mathcal H^n(A \cap T_x X)) \; d\mathcal H^n (x) \; .$$
 We extend $\mathcal M$ to a measure on $TX$ by setting $\mathcal M(TX\setminus TX_{reg})$ to be $0$.
By definition, a subset $A\subset TX$ is $\mathcal M$-measurable if and only if there exists  a Borel subset $A'\subset A\cap TX_{reg}$ such that
for $\mathcal H^n$-almost all $x\in X$ the intersection $(A\setminus A') \cap T_xX$ has $\mathcal H^n$-measure zero in $T_xX$.




 For any $\lambda >0$ the multiplication with $\lambda$  on all $T_xX$ sends $\mathcal M $ to $\lambda ^n \cdot \mathcal M$.
 Consider the canonical involution $I\:TX_{reg}\to TX_{reg}$, defined by $I(v)=-v$.
This involution preserves $\mathcal M$, since it preserves the Lebesgue measure in each tangent space.
%We extend $I$ to an arbitrary Borel map outside of $TX_{reg}$ (for instance, $I(v)=v$ for all $v\notin TX_{reg}$).


\subsection{Geodesic flow}
   In order to define the geodesic flow $\phi$ on a maximal subset   $\mathcal F$ of $TX \times \R $  we proceed as follows.
 For any  $v\in  T_xX$ we set $\phi (v,0)=v$.
 If no geodesic starts in the direction of $v$,
 the value $\phi (v,t)$ will not be defined for $t\neq 0$. If such a geodesic $\gamma_v$ exists, then $\gamma_v$   can be uniquely extended to a  local geodesic $\gamma_v \:[0,a)\to X$  defined on a maximal possible interval, since  geodesics do not branch in $X$.  For $t\geq a$
 the value $\phi (v,t)$ will not be defined. For $0<t<a$ we set $\phi (v,t)$ to be $\gamma _v ^+ (t) \in T_{\gamma _v(t)} X$, the starting direction of $\gamma_v\:[t,a) \to X$ at $\gamma_v (t)$.

 Whenever, the geodesic $\gamma _v\:[0,a)\to X$ extends to an (again uniquely defined, maximal)  local geodesic   $\gamma_v\:(b,a) \to X$ for some $b<0$
 then we define $\phi (v,t)$ for $b<t<0$ to be $\gamma_v ^+ (t) $ as above.


For $\lambda >0$ and $v\in  T_x X$ we have  $\phi (\lambda v,t) :=\lambda \cdot \phi (v,\lambda t)$, whenever the right hand side is defined. Moreover,
 $\phi _t ( \phi _s (v))= \phi _{t+s} (v)$ whenever the left hand side is defined for $t,s\in \R$.
The partial flow $\phi$ preserves the norm of tangent vectors.  Due to \cite{Petparallel},  $TX_{reg}$ is invariant under this partial flow.
%On $TX_{reg}$ we have
%$I\circ \phi_t = \phi _{-t} \circ I$, whenever both sides are defined.



By construction, the geodesic flow is defined "at least on the domain of the definition of the exponential map".
More precisely, let $D=\cup _{x\in X} D_x$ be the set of all vectors $v\in TX$ for which $\exp _{\pi(v)} (v)$ is defined.
Then $D$ is invariant under multiplication with any $0\leq \lambda \leq 1$.  Moreover, for all $v\in D$ and all $\lambda <1$
the geodesic flow $\phi_1 (\lambda v)$ is defined and $\pi (\phi_1 (\lambda v))=  \exp _{\pi (v)} (\lambda v)$.
From this we deduce the following.
 For $\mathcal M$-almost all $v\in D$, we have $v\in D\cap TX_{reg}$, $\phi _1(v)$ is defined  and contained in $TX_{reg}$. Moreover,
 $w=-\phi_1 (v)$ is contained in $D$  and
\begin{equation} \label{eq:symm}
(\pi (w), \exp (w))=(\exp (v), \pi (v)) \in X\times X\; .
\end{equation}


\subsection{Measurability questions} \label{subsec:measur} 
 In this subsection we shortly discuss measurability question of the partial geodesic flow
$\phi:\mathcal F\subset TX\times \R \to TX$ constructed above.\footnote{A.:  Maybe skip or shorten this subsection it?{\color{red} Nothing interesting happens here so i think it's ok to skip it. Vitya}   Maybe one can put this subsection and the subsequent remark in one short remark, saying that 
measurability is easy to check?}
Consider an exhaustion  $X=\cup U_k$ by an increasing sequence of relatively compact open subsets $U_k$.  For any $k$, let $\mathcal I_k$ be the set of local geodesics $\gamma :I\to U_k$ with the following properties. The interval of definition $I$ is contained in $[- k , k]$  and the restriction of $\gamma$ to any subinterval of length $\frac 1 k$ is a (minimizing) geodesic. Then any sequence $\gamma _i$ in $\mathcal I_k$ has a convergent subsequence and the limit curve is a local geodesic.  Using this observation it is easy to see that the domain of definition $\mathcal F$
of $\phi$ can be represented as a union of subsets  $\mathcal F_k$  (starting directions and interval lengths of the curves in $\mathcal I_k$) with the following properties. The intersection  $\mathcal F_k \cap TX_{reg} \times \R$ is closed and the restriction $\phi:\mathcal F_k \cap TX_{reg} \to TX_{reg}$ is continuous.

Therefore, we deduce, that the intersection $\mathcal F\cap TX_{reg}$ is a Borel subset of $TX_{reg}$ and the geodesic flow
$\phi :\mathcal F\cap TX_{reg} \to TX_{reg}$ is a Borel map.

\begin{rem}  \footnote{A.: Delete or leave? }
Note that in general, even the intersections $\mathcal F\cap T_xX \times \R$ does not need to be open nor closed in $T_xX \times \R$.
The above argument shows that these intersections are always Borel subsets.
\end{rem}






\subsection{Liouville property}
We continue to use the above notations.
 Denote by $\mathcal G$ the set of all vectors $v\in TX $ such that $\phi (v,t)$ is defined for all $t\in \R$. By definition, $\mathcal G$ contains the $0$-section, it
 is invariant under multiplications by any $\lambda >0$  and it is invariant under the geodesic flow $\phi$. Moreover, $\mathcal G\cap TX_{reg}$ is invariant under the involution $I$.
  \begin{defn}
 We say that an Alexandrov space $X$ has the \emph{Liouville property} if  $TX\setminus \mathcal G$ has zero Liouville measure  and if
for any $t\in \R$ the geodesic flow $\phi _t \:\mathcal G\to \mathcal G$ preserves the Liouville measure.
 \end{defn}

 %  For a subset $K\subset X$ and a number $r<0$ we use the notation $T^ r K$
%to denote the set of all vectors in $\pi^{-1} (K)=\cup _{x\in K} T_xX$ of norm $< r$.
The Liouville property can be checked infinitesimally as follows.

\begin{lem} \label{infini}
An Alexandrov space $X$ does not have the Liouville property if and only if there is a compact subset $K\subset X$, a positive number
$\epsilon$ and a sequence of positive numbers  $r_m \to 0$  with the following property.
 For every  $m$, there exists a Borel subset $A_m\subset T^{r_ m} K$ such that
 \begin{equation} \label{eq:m}
\epsilon \cdot r_m^{n +1} \leq \mathcal M (A_m) -\mathcal M (\phi _1 (A_m)) \;.
\end{equation}
\end{lem}
By an abuse of notations we denote in \eqref{eq:m}  and below by  $\phi_1 (A_m)$ the set of all $\phi _1(v), v\in A_m$, for which
$\phi _1(v)$ is defined.


\begin{proof}
If at least one $r_m$ with the above property exists, then $X$  does not have the Liouville property by the very definition.

Assume on the other hand, that $X$ does not have the Liouville property. Then,  by homogeneity of the geodesic flow,  $\phi _1$ is not defined almost everywhere on $TX$ or it does not preserve the measure $\mathcal M$. Thus, we find a compact subset $K_1\subset X_{reg}$ and (by homogeneity) a Borel subset $A\subset T^1 K_1$ such that either $\phi_1$ is not defined on almost all of $A$ or $\mathcal M (A) > \mathcal M(\phi (A))$.  Replacing $A$
if needed by a smaller  subset, we find in both cases some $\epsilon >0$, such that   $$\epsilon < \mathcal M (A) - \mathcal M(\phi_1 (A)) \; .$$
Since $\phi_1 (A)=2\cdot \phi _2 \Big (\frac 1 2 \cdot A \Big)=2\cdot \Big(\phi _1 \circ \phi_1 \Big (\frac 1 2 A \Big ) \Big )$,
we obtain
$$\Big(\frac 1 2 \Big )^n \cdot \epsilon \leq  \mathcal M \big(\frac 1 2 A \big) - \mathcal M \big( \phi_1 (\frac 1 2 A ) \big) + \mathcal M \big(\phi_1 (\frac 1 2 A )\big)-
 - \mathcal M\big (\phi_1 \big (\phi _1 (\frac 1 2 A )\big) \big) \; .$$
Therefore,  either for $A_{\frac 1 2} := \frac 1 2  \cdot A$ or for $A_{\frac 1 2} := \phi_1 (\frac 1 2 \cdot A)$ we infer
$$\epsilon \cdot \Big(\frac 1 2 \Big )^{n+1}  < \mathcal M \big(A _{\frac 1 2} \big) - \mathcal M \big( \phi_1 (A _{\frac 1 2}) \big) \; .$$
Note that $A_{\frac 1 2}$ constructed above is contained in $T^{\frac 1 2} K_{\frac 1 2}$, where $K _{\frac 1 2}$ is the closed tubular neighborhood  of
radius $\frac 1 2$ around $K$.

Iterating the above procedure we obtain  for $r_m =\frac 1 {2^m}$   a subset $A_{r_m} \subset T^{r_m} K_m$  such that $K_m$ is the closed tubular neighborhood around $K_{m-1}$ of radius $r_m$ and such that \eqref{eq:m} holds true.  The claim follows now since all $K_m$ are contained in the closed tubular neighborhood of radius $1$ around $K_1$ and this set is compact, by completeness of $X$.
\end{proof}


\begin{rem}
The completeness of the Alexandrov space $X$ is used in the proof of \tref{thmmain} only once, namely in the last line of the above proof.
\end{rem}

\subsection{Relation with the mm-boundary}
Before proceeding to the proof of  \tref{thmmain}, we interpret the signed measures $v_r$ from \eqref{eq:first} in suitable geometric terms.
Let $K\subset X$ be   any measurable  subset $K\subset X$ and let $r>0$ be arbitrary.
Since $\mathcal H^n (X\setminus X_{reg} )=0$, we have
$$\mathcal M (T^r K) =\int _K \omega _n \cdot r^n  =\omega _n \cdot r^n \cdot \mathcal H^n (K) \; .$$
Denote now by $U^r(K)$ the set of all pairs $(x,y)\in X\times X$ with $x\in K$ and $d(x,y)<r$.
By Fubini's theorem the set $U^r(K)$ is $\mathcal H^n \otimes \mathcal H^n =\mathcal H^{2n}$ measurable and we have
$$\mathcal H^{2n} (U^r (K))= \int _K b_r (x) \;  d\mathcal H^n (x) \;. $$
Taking both equations  together we see that the measure $v_r$ expresses exactly the difference between both measures. More precisely,
\begin{equation} \label{eq:compare}
v_r (K) = \frac 1 {r^n} \Big(\mathcal M (T^r K)- \mathcal H^{2n} (U^r (K)) \Big)\;.
\end{equation}



Now we can prove the following result,  reformulating \tref{thmmain}.
\begin{thm} \label{reform}
If the Alexandrov space $X$ has trivial mm-boundary then  it has the Liouville property.
\end{thm}

\begin{proof}
Assume that $X$ does not have the Liouville property and consider the compact subset $K \subset X$, the positive numbers $\epsilon, r_m$ and the Borel
subsets $A_m\subset T^{r_m} K$ provided by \lref{infini} above.

Let $Y$ be the closed tubular neighborhood  of radius $1$ around $K$.
Let again $D\subset TX$ be the set of all vectors at which the exponential map is defined.
  For $r>0$, denote  by $N^r$ the intersection of $D$ with $T^r Y$ and consider the "total exponential map"
$E\:N^r \to X\times X$ given by $E(v)= (\pi (v), \pi (\exp (v)))$.   Let $U^r =U^r(Y)$ be as above the set of all
  pairs $(y,x) \in X\times X$ with $y\in Y$ and $d(x,y)<r$. Then
\begin{equation} \label{eq:image}
E(N^r) =U^r \; .
\end{equation}
Moreover, for any  fixed $x \in Y$, the restriction of $E$ to $D_x \cap N^r$ is
a $(1+ C r^2)$-Lipschitz continuous map from $D_x\subset T_xX$ onto the set  $U^r \cap \{x \} \times X$.
Thus, for all sufficiently small $r$, and any Borel subset $S\subset D_x \cap N^r$, we have
$\mathcal H^n (E(S)) \leq (1+4n C r^2) \mathcal H^n (S)$.  Using the definition of the measure $\mathcal M$ and  Fubini's formula for the product measure
$\mathcal H^{2n} =\mathcal H^n \otimes \mathcal H^n $ on $X\times X$ we obtain for any $\mathcal M$-measurable subset $S$ of $N^r$
\begin{equation} \label{eq:contract}
\mathcal H^{2n} (E(S)) \leq (1+4nC r^2) \cdot \mathcal M(S)\; .
\end{equation}
Due to \eqref{eq:compare}, the triviality of the mm-boundary of $X$ implies
$$\lim _{r\to 0} \frac 1 {r^{n+1} } |\mathcal M(T^r Y) -\mathcal H^{2n} (U^r)| =0 \;.$$
	Together with  \eqref{eq:contract} this implies that for every  $\delta >0$, there exists some $s>0$ with the following property. For all $0<r<s$
	we have
	\begin{equation}  \label{eq:almostall}
	\mathcal M (T^r Y)- \mathcal M (N^r) < \delta r^{n+1} \; ,
	\end{equation}
	and  for all measurable subsets $S\subset N^r$ we have
	\begin{equation}  \label{eq:almostiso}
	|\mathcal H^{2n} (E(S)) - \mathcal M  (S)|  < \delta r^{n+1} \;.
\end{equation}
For any measurable subset $S\subset N^r \cap TK$  we now claim
\begin{equation} \label{eq:finally}
|\mathcal H^{2n} (E(S)) - \mathcal M (\phi _1 (S))|  <  2 \delta r^{n+1} \; .
\end{equation}
In order to prove \eqref{eq:finally}, let $S^+$ be the subset of all vectors $v\in S $ for which $\phi_1 (v)$ exists  and is contained in $TX_{reg}$.
Then $ \mathcal M (S\setminus S^+ )=0 $, hence $|\mathcal H^{2n} (E(S))- \mathcal H^{2n} (E(S^+))|< \delta r^{n+1} $, due to  \eqref{eq:almostiso}.
 By definition, $\phi _1 (S\setminus S^+)\cap TX_{reg} =\emptyset$, therefore $\mathcal M (\phi _1 (S\setminus S^+)) =0$.


For all $v\in S^+$, we have $-\phi _1 (v) \in N^r$ and,   due to \eqref{eq:symm},
$$E(-\phi_1 (v)) =J(E(v)) \;.$$
The involution  $I(v)=-v$ is $\mathcal M$-preserving on $TX_{reg}$.  And the involution
 $J\:X\times X\to X\times X$ given by $J(x,y)=(y,x)$ preserves $\mathcal H^{2n}$. Therefore, from \eqref{eq:almostiso} we deduce
 $|\mathcal H^{2n} (E(S^+)) - \mathcal M (\phi _1 (S^+))|  <   \delta r^{n+1} $.
  This finishes the proof of \eqref{eq:finally}.


Coming back to our subsets $A_m\subset T^{r_m} K$ we have
 $$\epsilon \cdot r_m^ {n+1} \leq \mathcal M ( A_m) - \mathcal M( \phi _1 (A_m)) \leq \mathcal M ( A_m) - \mathcal M( \phi _1 (A_m\cap N^r)) \,.$$
Setting  $S_m=A_m\cap N^{r_m}$ we estimate the right hand side from above by
$$|\mathcal M (A_m) -  \mathcal M(S_m)|
+|\mathcal M (S_m )  - \mathcal H^{2n} (E(S_m) )| + | \mathcal H ^{2n} (E(S_m)) - \mathcal M (\phi _1(S_m)|  \;.$$
Applying  \eqref{eq:almostall}, \eqref{eq:almostiso} and \eqref{eq:finally}  this expression is bounded above by $4 \delta r _m^{n+1}$, for all $m$ large enough. Taking the inequalities together we obtain,  $\epsilon \cdot r_m^ {n+1}< 4 \delta r _m^{n+1}$ for all $m$ large enough.
Since $\delta$ is arbitrary small, this provides a  contradiction  and  finishes the proof of \tref{reform} and therefore of \tref{thmmain}.
\end{proof}









%\subsection{Another measure}

%We will need a simple lemma:

%\begin{lem} \label{anothermeasure}
%Let $L,n$ be given. There is some $\bar L= \bar L (L,n)$ with the following properties.
%Assume that $Lr^n \geq \vol (B_r (x)) \geq  1/L r^n$ for all $x \in X$ and all small
%$4r\leq r_0$.  Let $K$ be a positive measure on $X$. Let $A\subset X$ be given, with finite $K (B_{2r} (A))$.
%Then $\int _A K(B_r (x)) d\vol (x) \leq \bar L r^n \cdot K (B_{2r}(A))$.
%\end{lem}

%\begin{proof}
%Cover $A$ by balls $B_i$ of radius $2r$, such that the  balls with the same centers and radii $r$  still cover $A$ and such that
%the intersection multiplicity of the covering $B_i$ is bounded by some
%$L'=L' (L)$. Then estimate  all $K(B_r (x))$ by $K(B_i)$ for some  for some $i$,
%with $x\in B_i$. We deduce

%$$\int _A K(B_r (x)) d\vol (x) \leq \int _A  K(B_i) \cdot Char _{B_i} (x)  d\vol (x) \leq $$

%$$\leq \Sigma  K(B_i) L (2r)^n \leq 2^n L L' K(B_{2r} (A))$$
%\end{proof}


%\subsection{Regular-Singular Decomposition}

%\begin{prop} \label{regsing}
%Let $X$ be a space with another positive measure $K$ (absolute value of scalar curvature). Assume that there are constants
%$C$ (measure of singular locus), $L$ (some growth constant), $l$ (equal to 1 or 2 for us), $\delta ,r_0$
%(some thresholds) such that the following holds true.  For any $r<r_0$ there is a decomposition
%$X=X_r \cup Y_r$ with $\vol (Y_r) \leq C\cdot r^l$. For any $x\in X_r$ and any  $s\leq r$ the volume of
%$ (B_s (x)) $ is bounded by $L^{\pm 1} s^n$ and for any $x\in X$ it is bounded from above by $L\cdot s^n$.
%  Assume further, that for any  $x\in X_r$ the inequality
%$K(B_{Lr} (x)) \leq \delta  r^{n-l}$ implies $|\vol (B_r (x) ) - b_n r^n| \leq L \cdot K ( B_{Lr} (x)) \cdot r^l$.

%Then there is some number $\bar L$, such that for all small $r$ and all $A\subset X$ we have
%$\|v_r (A)\| \leq \bar L \cdot (K(B_{2Lr} (A)) +C) \cdot r^l$.
%\end{prop}




%\begin{proof}
% Set  $S^r := \{ x\in A \cap X_r| K(B_{Lr} (x)) > \delta  r^{n-l} \}$.
% Let $N_1$ be the maximal $2Lr$-net in $S^r$. Then $|N_1| \leq K (B_{Lr} (A)) /(\delta _0  r^{n-l})$ elements.
% Thus $\vol (S^r)$ is bounded from above by $2^n \cdot L ^{n+1} \cdot \delta _0 ^{-1} \cdot r^l \cdot K(B_{Lr} (A)) $.
% Therefore $|v_r |(S^r) \leq \bar L_1 \cdot K (B_{Lr } (A)) \cdot r^l$.

% On the other hand, $\vol (A \cap Y_r) \leq C \cdot r^l$.  Thus
% $|v_r| (A\cap Y_r) \leq \bar L_2 \cdot C \cdot r^l$.

% Finally, for any $x$ in $A^+ := A\setminus (S^r \cup  Y^r)$, we have
% $|\vol (B_r (x))- b_n r^n | \leq L \cdot K(B_{Lr} (x)) \cdot r^l$.

% Thus $r^n \cdot |v_r| (A^+) \leq \int _{A^+} L \cdot K(B_{Lr} (x)) \cdot r^l d\vol (x)
% \leq \bar L_3 \cdot r^{n+l} \cdot K (B_{2Lr} (A^+))$, by \lref{anothermeasure}.

% Summing up, we obtain the desired inequality.
%  \end{proof}

%\subsection{Limiting case}
%Under the assumptions of the previous section, let $\delta$  be fixed. Assume that as $r$ goes to $0$, all subsets $X_r$ are open (this assumption can be easiely achieved by enlarging $X_r$ slightly). Assume further that $X_r \subset X_s$ for
%all $s\leq r$. Set $X_{reg} = \cup X_r$ and $Y_0 = \cap Y_r$.  Set, finally,
%$C''=\limsup _{r\to 0}  \frac {vol (Y_r)} {r^l}$.

% From \pref{regsing}, we conclude that the measures $\frac {v_r } {r^l}$ are
%uniformly bounded.  Let $\mu$ be any weak limit
%$\mu = \lim _{r_i \to 0} (v_{r_i}  / r_i ^l)$.  Then for some constant
%$L$ depending on the privious $L$, the measure $\mu$ satisfies the following:

%1) $\mu (Y_0) \leq L C'$;

%2) $\mu$ is bounded on $X_{reg}$ by $L \cdot |K|$.


\subsection{Quasi-geodesics} 
Finally, we discuss some relations with quasi-geodesics, referring the reader to \cite{Petsemi} for the basic properties of such curves.
Recall, that whenever a unit speed geodesic $\gamma _v :[0,a] \to X$   start at a point $x$ in the direction $v$ then this is the unique quasi-geodesic defined on the interval $[0,a]$, \cite{PP}, p.8, thus the same statement is also true for local geodesics $\gamma _v$.

As in Subsection \ref{subsec:tb} we have a canonical measure $\mathcal M _1$   on the unit tangent bundle $\Sigma X \subset TX $ of $X$, which we also call the Liouville measure. Whenever
the Alexandrov space has the Liouville property, then the geodesic flow is defined $\mathcal M_1\otimes \mathcal H^1$-almost everywhere on $\Sigma X \times \R$ and preserves $\mathcal M_1$.  In this case    for $\mathcal M_1$-almost each unit direction there exists exactly one quasi-geodesic starting in this direction.

Let now $X$ be an  Alexandrov space with topological boundary $\partial X$ and let $Z$ be the doubling $X\cup _{\partial X} X$, which
is an Alexandrov space without boundary, \cite{P2}. Quasi-geodesics in $X$ are exactly the projections of the quasi-geodesics in $Z$ under
the canonical folding $f:Z\to X$.   From this we see that if $Z$ has the Liouville property, then $\mathcal M_1$-almost each  direction $v\in \Sigma X$ is the starting direction of a unique infinite quasi-geodesic and that the corresponding quasi-geodesic flow preserves $\mathcal M_1$.

Finally, as an application of \tref{thmmain} and \tref{alexandrovthm} we see that the above assumptions are fulfilled whenever the complement $X\setminus \partial X$ has trivial mm-boundary. Indeed, in this case the mm-boundary of $Z$ must be concentrated on $\partial X$, hence it must be trivial by \tref{alexandrovthm},(3).

Using that a limit of quasi-geodesics is a quasi-geodesic it is not difficult to conclude that the partial geodesic flow $\phi:\mathcal F\cap TX_{reg} \to TX_{reg}$ defined above is continuous on the whole domain of definition, slightly strengthening Subsection \ref{subsec:measur}.



\section{Surfaces with bounded integral curvature in the sense of Alexandrov} \label{sec:surface}
\subsection{Preparations} We assume here some  familiarity  with the theory of surfaces with bounded integral curvature, cf. \cite{AZ} and
 \cite{Reshetnyak-GeomIV}.  Such a surface $X$ is a locally geodesic metric space, homeomorphic to a two-dimensional surface and it comes along with a
signed Radon measure, the so called curvature measure $\Omega$, \cite{Reshetnyak-GeomIV}, Section 8. Unlike when talking about Alexandrov spaces \emph{we will not assume that $X$ is complete}.
 Consider the total variation $|\Omega |$ of $\Omega$.
 We will derive  \tref{intsurface} as a consequence of the following local and non-exact version of a theorem of M. Bonk and U.Lang, \cite{Bonk-Lang}, relating the curvature measure $\Omega$ to the volume of balls.

\begin{lem} \label{lem:bl}
There exists some $\delta _0>0$ with the following property.
Let $X$ be a surface with bounded integral curvature and let  $\Omega\in \mathcal M(X)$ be its   curvature measure. Let $x\in X$ be a point and let $r>0$ be such that
$\bar B_{r} (x)$ is compact,  $| \Omega| ( B_{r} (x) )  < \delta _0$ and $\bar B_{r} (x)\subset U$ where $U$ is an open subset of $X$ homeomorphic to $\R^2$.
Then $$|1- \frac {b_r(x)}  {\pi r^2} | \leq 5 \cdot |\Omega |( B_{r} (x)) \; .$$
\end{lem}


\begin{proof}
Set $\delta = |\Omega |( B_{r} (x))$. By continuity, it is sufficient to prove that $|1- \frac {b_s(x)}  {\pi s^2}| \leq \delta $ for any $s<r$.  Using approximations of
the metric on $X$ by polyhedral metrics \cite[Theorem 8.4.3, Theorem 8.1.9]{Reshetnyak-GeomIV},  it is sufficient to prove the claim  for polyhedral metrics. Thus we assume from now on that $X$ is polyhedral.

We claim  that there exists a complete polyhedral surface $\hat X$ homeomorphic to a plane,  which contains a copy of $B_s (x)$ and such that
the curvature measure $\hat \Omega$ of $\hat X$  satisfies $|\hat \Omega | (\hat X) \leq 4\delta$.  Once the claim is proven  and $4\delta < 2\pi$,
\cite{Bonk-Lang} provide us a biLipschitz map $f:X_0\to \R^2$ with biLipschitz constant at most $1+\frac {4\delta} {2\pi -4\delta}$.
For sufficiently small $\delta$, we would only need to apply \eqref{eq:bilip} to finish the proof of the lemma.

It remains to prove the existence of $\hat X$. We take some $r>t>s$ and consider the compact  metric ball $\bar B_t (x) \subset B_r (x)$. For simplicity, we may assume that  the boundary $S_t=S_t (x)$ of
$\bar B_t (x)$ does not contain singular points of $X$. Due to \cite[Theorem 9.1, Theorem 9.3] {Reshetnyak-GeomIV}, the boundary $S_t$ is a (piecewise smooth) Jordan curve, once $\delta _0< 2\pi$, and the negative part $\kappa ^-$ of the geodesic curvature $\kappa$ of $S_t$  satisfies $|k^-| (S_t)  \leq \delta$.  Since $U$ is homeomorphic to a plane this implies that $\bar B_t (x)$ is homeomorphic to a closed disk $\bar D^2$ in $\R^2$.


We find a polygonal Jordan curve $\Gamma$ in $B_t(x)$  approximating $S_t$  such that the negative part of the geodesic curvature of $\Gamma$ is smaller than $2\delta$.
Now we glue to any edge of $\Gamma$ a flat half-strip.   At vertices of $\Gamma$ at which $\Gamma$ encloses an angle at least $\pi$ we glue the corresponding rays together.   At other vertices we glue in addition to such a vertex a sector so that the total angle at such a vertex will be $2\pi$.  The arising space $\hat X$  is a complete polyhedral plane. All of its singularities are contained in $\bar B_t(x)$  and the curvature measure $\hat \Omega$  of $\hat X$ satisfies $|\hat \Omega | ( \hat X) = |\Omega (B_s (x))| + |\hat \Omega | ( \Gamma )$. Moreover, by construction  the last summand is bounded from above by $|\kappa ^-| (\Gamma )<2\delta$.
Lastly, since  $\bar B_t (x)$ is homeomorphic to $\bar D^2$ it follows that $\hat X$ is topologically a plane.

   This finishes the proof of the claim and of \lref{lem:bl}.
\end{proof}



 \subsection{Local finiteness of mm-curvature}
 Now we are in position to prove
 the following generalization of  \tref{intsurface}:
 \begin{thm}\label{intsurface1}
 Let $X$ be an Alexandrov surface with integral curvature bounds.  Then $X$ has locally finite mm-curvature.
  \end{thm}
 %
 \begin{proof}[Proof of \tref{intsurface1}]
 Let again $\Omega$ denote the curvature measure of $X$ and let $\delta _0>0$ sufficiently small and satisfy the conclusion of  \lref{lem:bl} above.
 The statement  of \tref{intsurface} is local, so we need to prove it only in a small neighborhood of any point.
 Thus we may and will assume that there is a point $x_0\in X$
 such that $| \Omega| (X \setminus  \{ x_0 \} )< \delta _0$ and  $X$ is homeomorphic to a plane.



  Let now $A\subset X$  be a compact subset. Choose some $\epsilon >0$ such that the closure of  $B_{3\epsilon} (A)$ in $X$ is compact and such that,
  for any $0<3r<\epsilon$ it holds that  $\mathcal H^2 (B_{3r}(x_0)) < \frac 1 {\epsilon} r^2$, cf. \cite[Lemma 8.1.1]{Reshetnyak-GeomIV}.

 Let  $r<\epsilon$    be arbitrary.
 For any $x\in B_{2r} (x_0)$ we have $b_r (x)  =\mathcal H^2 (B_r (x)) \leq \mathcal H^2 (B_{3r} (x_0)) \leq \frac 1 {\epsilon} r^2 $.


For any $x\notin B_{r} (x_0)$ we have $| \Omega |(B_r (x)) < \delta _0$. Therefore, in this case  $|1-\frac {b_r(x)}  {\pi r^2} | \leq 5 |\Omega | ( B_{r} (x))$ by Lemma~\ref{lem:bl}.

For the measures $v_r$ from \eqref{eq:first} we estimate:
 $$|v_r| (A\cap B_{2r} (x_0))) \leq |v_r| (B_{2r} (x_0))  \leq (1+ \frac 1 {\epsilon}) \cdot \mathcal H^2 (B_{2r} (x_0)) \leq
 (1+ \frac 1 {\epsilon}) \cdot \frac 1 {\epsilon} r^2 \; .$$
On the other hand,
 %for any $x\in A\setminus B_{r} (x_0)$ we have $\bar \Omega (B_r (x)) < \delta _0$, hence $|\frac {b_r(x)}  {\pi r^2}  -1| \leq 5 |\Omega | (\bar B_{r} (x))$.  Therefore,
$$|v_r| (A\setminus B_{2r} (x_0)) \leq \int _{A\setminus B_{2r} (x_0)} 5 \cdot |\Omega| (B_r (x)) \, d\mathcal H^2 (x) \leq 5 \int _Q \mathcal H^2 (B_r(x))  \, d|\Omega | (x) \; ,$$
where $Q=B_r(A\setminus B_{2r} (x_0)) \subset X\setminus \{ x_0 \}$, and where we have used \lref{lem:exchange} in the last step.
 For any $x\in Q$ we have $\mathcal H^2 (B_r(x))  =b_r (x) \leq 2\pi  r^2$, once $\delta _0$ has been chosen sufficiently small.
We deduce $|v_r| (A\setminus B_{2r} (x_0)) \leq 10 \pi \delta _0 \cdot r^2$.

Altogether we see that $|v_r| /r^2$ is uniformly bounded on $A$, independent of $r$.  This finishes the proof.
\end{proof}









\section{Convex hypersurface} \label{sec:hyper}
We are going to prove \tref{thmfirst} in this section. The proof will follow from \tref{thmmain} by comparing the mm-boundary with the mean curvature measure on such convex hypersurfaces.
It is possible to slightly refine and extend  the arguments below and to deduce the theorem without a reference to \tref{alexandrovthm}, but using  our main technical result \tref{alexandrovthm}
shortens the proof.
\subsection{Mean curvature}
Let $X^n$ be a  convex hypersurface in $\R ^{n+1}$.  Recall that there exists a Radon measure $K$ on $X$, called the mean curvature measure, which has the following properties, cf. \cite{Schneider}, \cite{Fedcurvature}.
For smooth hypersurfaces $X$ the mean curvature measure is the multiple of the usual mean curvature function and $\mathcal H^n$.  The mean curvature measure is stable under Hausdorff convergence of convex hypersurfaces in $\R^{n+1}$. If the hypersurface is rescaled by $\lambda$, the mean curvature $K$ is rescaled by $\lambda ^{1-n}$.

We call a point $x$ in the convex hypersurface $X$ a smooth point if there exist a unique supporting hyperplane of $X$ through this point.
Due to stability of the mean curvature measure, we get by rescaling the following conclusion.
%For any such point $x\in X$ and any sequence $x_i$ converging to $x$ in $X$ blow ups of $X$ centered at $x_i$ converge to a flat hyperplane.
For any smooth point $x\in X$ and any $\epsilon >0$, there exists some $t>0$, such that
for any $y\in B_{t} (x)$ and any $0<r<t$ we have $K(B_r(y))  \leq \epsilon \cdot r^{n-1}$.








The following lemma is basic:

\begin{lem}  \label{lem:mean}
There exist numbers $\delta _0,C>0$ depending only on $n$ with the following property. Let $X$ be a convex hypersurface in $\R^{n+1}$.
Let $x\in X$ be a point and $r>0$ be such that the mean curvature $K$ satisfies  $K(B_{2r} (x))  \leq \delta \cdot r^{n-1}$ with $\delta <\delta _0$.
Then $$| 1 -  \frac {b_r (x)} { \omega _n r^n}| <C\cdot \delta \cdot K( B_{2r} (x)) \cdot r \; .$$
\end{lem}


\begin{proof}
By resclaing it suffice to prove the existence of $\delta _0, C>0$ such that the lemma holds true for $r=1$.
By approximation it is enough to prove the result for smooth convex hypersurfaces.  Since the mean curvature vanishes only
if $X$ is a flat hyperplane, using the stability of $K$ under convergence, we may assume that  for a fixed number $\frac 3 2<s<2$ the ball $U=B_{s} (x) \subset X$ is arbitrary close
to a flat hyperplane in $\R^{n+1}$, once $\delta _0$ has been chosen small enough.  Thus the tangent hyperplanes  to points in $U$  can be assumed to be sufficiently close to the the tangent
space  $W =T_xX \subset \R^{n+1}$. Thus we may assume that $U$ is given as a graph  of a convex function $f:V\to \R$ defined on an open subset $V\subset W$.
Moreover, by choosing $\delta _0$ small enough we may assume
that the norm of the  gradient of $f$ is everywhere smaller than any fixed positive number $\epsilon >0$.
The orthogonal projection $P:U \to V$ is biLipschitz and the biLipschitz constant is bounded from above by $\sqrt {1+ |\nabla f|^2} \leq 1+ |\nabla f| ^2 <1+\epsilon ^2$.

Therefore, applying \eqref{eq:bilip} we only need to verify the following claim.  Let $O$ be the ball of radius $\frac 3 2$ around $0$ in $\R^n$. Let $f:O\to \R$ be a convex  smooth function
with $f(0)=|\nabla f (0)|=0$. Assume that the convex hypersurface $Z= \{(x,f(x)) | x\in O \}$ has total mean curvature $\delta <\delta _0$, where $\delta _0$ is sufficiently small.
Then the norm of the gradient of $f$ is bounded on $B_1 (0)$ by $L\cdot \delta$ for a constant $L$ depending only on the dimension $n$.
\end{proof}




\subsection{The proof} Now we can provide the proof of
\begin{thm} \label{thmconv}
Let $X=X^n$ be a convex hypersurface in $\R^{n+1}$.  Then it has trivial mm-boundary.

\end{thm}
Together with \tref{intsurface} this finishes the proof of \tref{hypersurface}. In combination with \tref{thmmain} it also finishes the proof of \tref{thmfirst}.
% it is sufficient to show that $X$ has trivial mm-boundary. This claim is local on $X$.
%Taking a small neighborhood of an arbitrary point in $X$ we can extend it to compact convex hypersurface. Thus we may and will assume that $X$
%is compact.
\begin{proof}
%For any $r>0$ the signed measure $v_r$ on $X$ defined by \eqref{eq:first} is non-negative by the theorem of Bishop-Gromov.
 Since $X$ has locally finite mm-boundary by
\tref{alexandrovthm}, it suffices to prove that  any limit measure $\nu$ of a sequence $\frac 1 {r_n} v_{r_n}$ for $r_n\to 0$ must be the zero measure.
Let us fix such a limit measure $\nu$.  Again due to \tref{alexandrovthm}, $\nu (A)=0$ for any Borel subset $A\subset X$ with $\mathcal H^{n-1} (A)<\infty$.
Let $Y\subset X$ be the set of smooth points of $X$.
%i.e. of points with exactly one supporting hyperplane.
The complement $X\setminus Y$ is a countable union of subsets with finite $(n-1)$-dimensional Hausdorff measure (see \cite{Schneider}, Theorem 1.4), therefore $\nu (X\setminus Y) =0$.


It suffices to prove the following claim. For any $\epsilon   >0$ and  any point $y\in Y$ there exists some $t_0>0$ such that
for all $t<t_0$
\begin{equation} \label{eq:compari}
 \nu (B_t (y)) \leq \epsilon \cdot K (B_{4t}(y)) \; .
\end{equation}
Indeed, if \eqref{eq:compari} holds true, then, using the continuity and scaling properties of $K$ we deduce
$\nu (B_t(y)) \leq 2\cdot 4^{n-1} \cdot \epsilon \cdot K (B_t (y))$ for all sufficiently small $t$. Then by classical measure theoretical results, cf. \cite{Federer}, $\nu \leq 2\cdot 4^{n-1} \cdot  \epsilon  \cdot K$. Since $\epsilon$ was arbitrary, this would imply $\nu  =0$.

In order to prove the claim, we consider a sufficiently small $\delta$ smaller than  $ \delta _0$ from \lref{lem:mean}. We use the smoothness of the point $y$
to find some $t_0>0$ such that  for all $r \leq 4t_0$ and all
 $x\in B_{4t_0} (y)$ we have
$K(B_{2r} (x)) < \delta \cdot r^{n-1}$.    Due to  \lref{lem:mean}, for all $r<t<t_0$ we can estimate
$$v_r (B_t (y)) \leq \int _{B_t(y)} C \cdot  \delta \cdot r\cdot  K(B_{2r} (x))\, d\mathcal H^n (x) \leq \delta \cdot C \cdot r\cdot \int _{B_{3t} (y)}
\mathcal  H^n (B_{2r} (x)) \, dK (x) \; ,$$
where we have used \lref{lem:exchange} in the last step.
Since $\mathcal  H^n (B_{2r} (x)) \leq \omega _n \cdot (2r)^n$ by the theorem of Bishop-Gromov, we conclude
$$ v_r (B_t (y)) \leq  \delta \cdot C \cdot \omega _n \cdot  2^n \cdot r \cdot K (B_{3t} (y)) \; .$$
Dividing both sides by $r$ and letting $r$ go to $0$ we deduce \eqref{eq:compari}, once we have chosen $\delta$ such that
$\delta \cdot C\cdot \omega_n \cdot 2^{n+1} < \epsilon $.
\end{proof}








%\subsection{Generalization}
%The following more general statement seems to follow in the same way. Maybe it should be better steted after the section about
%BV-metrics.

%\begin{defn}
%A $DC$-submanifold $N$ of a smooth submanifold $M$ is a subset that can be locally around each point represented as a graph
%$N=(x,f(x))$ for a $DC$-map $f\:\R ^m \to \R ^n$.
%\end{defn}

%Main examples are $C^{1,1}$ submanifolds and convex hypersurfaces in Riemannian manifolds.
%Using the theorem of Nash, one can (isometrically) consider any $DC$-submanifold of a Riemannian manifold
%as a $DC$-submanifold of a euclidean space.

% It seems to me that everything explained above workes well in this more general situation and one obtains:

% \begin{prop}
% Let $N$ be a $DC$-submanifold of a smooth Riemannian manifold. Then $N$ has trivial mm-boundary.
% \end{prop}


\section{BV-estimate}\label{sec-BV-estimate}
\subsection{The smooth case}

\begin{prop}\label{prop-smooth}
Let $U\subset \R^n$ be an open subset with a smooth Riemannian
metric $g$. Let $g_0$ denote the flat background metric and assume $\|g_x-g_0\| \leq \epsilon$ for some fixed small
$\epsilon$.   Let $O\subset U$ be given, such that $B_{2r} (O)$ is compact in $U$.
Then for any $A\subset O$ we have $v_r (A) \leq c_n \cdot r \cdot \int _{B_{2r} ( O)} \|g'\|$.
\end{prop}



 \begin{proof}
 We are going to prove a slightly more precise estimate, showing what parts of $\|g'\|$ are essential in the estimate.


 Set $K_v (x) = D_x g(v,v) (v)$.  If $\||v\|_0 -1| \leq 100\epsilon$  then

 $\lim _{t\to 0} \frac {\||v\|_{x+tv} - \|v\| _x|} t \leq |K_v (x)|$.

 We set $K(x)= \Sigma |K_{v_i}(x)|$, where $v_i$ runs through a set of $n^3$ unit vectors in general position.
 Then $K_v(x) \leq K (x)$, for all unit vectors $v$.

 Set $B^x = \{ y | \|y-x\| _{g_x} \leq r \}$ and $B_x = B_r (x)$.  Let $\vol ^x$ be the constant
 volume form coresponding to the constant flat metric $g_x$.
   Set $C_x = B^x \setminus B_x$.

   The volume of $B^x$ with respect to $\vol ^x$ is equal to $b_n r^n$. Thus we only need to estimate
  the volume of $C_x$ and the difference between $\vol ^x$ and $\vol$.

 For a vector $v$ of norm close to $1$ and $x\in O$ we set
 $l_v ^x = L ([x,x+rv]) - r \cdot \| v\| _x$.

 \begin{lem}
 $|l_v ^x| \leq \int _0 ^r  \int _0 ^t | K_v (x+sv)| ds dt$.
 \end{lem}


 \begin{proof}
 $\||v\|_{x+tv} -\|v\| _x| \leq \int _0 ^t  |K_v (x+sv)| ds$. The result follows by integration.
  \end{proof}


   Now we have

   $\vol ^x (C_x) \leq \int _{S^{n-1}} r^{n-1} l^x _v d(v) $.


    We deduce

    $\int _O vol ^x (C_x) d\vol (x) \leq r^{n-1} \int _O (\int _{S^{n-1}} l_v ^x dv) d\vol (x) \leq \\
    r^{n-1} \int _O (\int _{S^{n-1}} (\int _0 ^r (\int _0 ^t |K_v (x+sv)| ds)dt)dv)d\vol (x) \leq \\
    r^n \int _O (\int _{S^{n-1}} (\int _0 ^r |K_v (x+sv)| ds) dv)d\vol (x)$.


    \begin{lem}
    For a non-negative function $u\in L^1$ we have the inequality
    $\int _O \int _{S^{n-1}} \int _0^r u(x+sv) ds dv dx \leq C \cdot r \cdot \|u\| _{L^1}$.
    \end{lem}


     \begin{proof}
     The left hand side is equal to

     $C_1 \int _O \int B_r (x) \frac {u(y)}  {\|x-y\| ^{n-1}} dx dy = C \int _O \int _O \frac {u(y) \xi _{B_r (x)}}
     {\|x-y\| ^{n-1}} dx dy \leq C_2 r  \int _O (u(y))$
     \end{proof}


      Thus we obtain $\int _O (vol _x (C_x)) \leq C \cdot r^{n+1} \|K\| _{L^1}$.


      It remains to estimate $\vol ^x - \vol$. We have

      $|\vol ^x (B^x) -\vol (B^x)| \leq \int _ {B^x} |\vol ^x -\vol ^y| dy = \\
      \int _{S^{n-1}} \int _0 ^r  t^{n-1} | \vol _{x+tv} - \vol _x| dt dv \leq  \\
       \int _{ S^{n-1}} (\int _0 ^r t^{n-1} (\int _0 ^t  |D_{x+sv} \vol (v)| ds) dt) dv$.

      Now the derivative $D_x \vol (v)$ can be estimated form above by $\epsilon \|D_xg (v)\| + | D_x s (v)|$,
      where we let $s(x)$ be the trace of $g(x)$ with respect to the background metric $g_0$.

      Integrating over $O$ and applying the above lemma two more times we obtain
      $v_r (O) \leq q_1 +q_2 +q_3$, where
      $q_1 = C r \|K\| _{L^1}$, $q_2 = \epsilon C \| Dg\| _{L^1} $ and $q_3 = C \|Ds\|$.

      This implies the claim.
    \end{proof}


\begin{rem}
Estimating  $v_r (A)$ from below seems to be more difficult.  We do not need a lower estimate  in the general case in this paper because in our applications $v_r$ is nonnegative due to Bishop-Gromov volume comparison. \footnote{ {V.:\color{red} do we need to go into this?} A.: No it is not needed. Should we mention the problem somewhere or just delete it?  }
\end{rem}

\subsection{DC- functions and DC-maps}
before proceeding let us briefly recall some known facts about $DC$ functions and $DC$ maps.
Let $U$ be an open subset of $\R^n$.
Recall that a function $f\co U\to \R$ is called $DC$ if it can be locally represented as  $h-g$ where $h$ and $g$ are locally Lipschitz and semiconcave.


A map $f=(f_1,\ldots, f_m)\co U\to R^m$ is $DC$ if each of its coordinates is $DC$.
 $DC$ functions form an algebra and composition of two $DC$ maps is $DC$. Also, if $f,g$ are $DC$ functions and $g$ is never $0$ on $U$ then $\frac f g$ is $DC$.
Partial derivatives of $DC$ functions are $BV$. Second partial derivatives of $DC$ functions are signed Radon measures. Just like semiconcave functions, $DC$ functions have second differential  a.e. and their partial derivatives which exist a.e. are differentiable a.e.



\subsection{The non-smooth case}
Let $(M,R)$ be an $n$-dimensional  $DC^1$ manifold, i.e., the atlas is $DC$ and it is
$C^1$ on the subset $R$.  Assume that $\mathcal H^{n-1} (M\setminus R) =0$.  Let $g$ be a $BV^0$
Riemannian metric on $(M,R)$. This means, $g$ is a $C^0$ metric on $R$ and $g_{ij}$ are $BV$ in any chart in the $DC^1$ atlas.  We assume that $g$ is bounded  away from $0$ and $\infty$ in any chart.

 The tensor $g$ defines a metric $d$ on the subset $R$, by measuring lengthes of curves contained in $R$.
Any other locally bounded metric $\tilde g$ on $R$ induces on $R$ another distance function locally bi-Lipschitz to $d$.
  If $\tilde g$ is the restriction of a smooth metric on $M$, then by the smallness of $M\setminus R$,
  the induced metric on $R$ coincides with the restriction of the inner distance on $M$.
   Thus  the completion of $d$   defines an inner  metric on $M$
  (denoted by $d$ as well), that is locally bi-Lipschitz to any smooth metric.



 \begin{lem}
 Let $(M,R,g,d)$ be as above and assume that $M$ is an open subset of $\R^n$. Let $U$ be a subset
 of $M$, such that the ball $B_{2r} (U)$ is relatively compact in $M$. Then
 there is a sequence of smooth metrics $g^k$ on $M$, such that for the induced metrics $d_k$ the follwoing holds true.

 \begin{enumerate}
 \item $g^k$ converges to $g$ pointwise on $R$,
 \item  The distance functions $d_n$ satisfy $\sup |d_k (x,y)- d(x,y)| \to 0$, where the supremum is taken
 over all $x,y \in U$ with $ d(x,y) \leq r$.
 \item For all $n$, we have $\int _U  \|(g^k_{ij} )'\| \leq 2\|(g _{ij})'\|(  B_r (U))$.
 \end{enumerate}
\end{lem}




\begin{proof}
Apply the standard mollifier construction for the coordiantes $g_{ij}$.
\end{proof}


  We will call a chart $U$ as above regular, if it is $(1+\epsilon)$ bi-Lipschitz to
  the background metric. We call $(M,R)$ regular, if any point in $M$ has a regular
  chart.

 From this we deduce
 \begin{cor}\label{cor-dc-vr}
 Let $M,U$ be  as in the last lemma. If $U$ is regular  then we have the following inequality for some
 universal constant $C$:

 $v_r (U) \leq C \cdot r \cdot (\Sigma _{i,j} \|g_{ij} '\| (B_{2r} (U))$.
\end{cor}


\begin{defn}
Let $(M,R,g)$ be a regular $BV^0$ Riemannian manifold. For any  subset
$A$ in $M$ we define $\mu (A)$ to be the infimum $\Sigma \|g '\| (U_i)$   over all finite open coverings $A\subset  \cup U_i$
by regular charts $U_i$.
\end{defn}

 Note that $\mu$ is a metric outer measure and defines a Borel measure $\mu$.


 The above observation gives us:
 \begin{cor}\label{cor-mu-dc}
 For any relatively compact subset $A$ of a regular $(M,R,g)$ and all sufficiently small $r \leq r_0 (A)$ the
 inequality $v_r (A) \leq C \cdot r \cdot \mu (A)$ holds true.
 \end{cor}





\section{Alexandrov spaces} \label{sec:Alex}


\subsection{Strained points}
Let $p$ be a point in an  Alexandrov space $X^n$ of $\curv\ge k$. Recall that $p$ is called \emph{$(m,A,\delta)$-strained} if there exist points\\ $a_1,\ldots, a_m, b_1,\ldots, b_m$ such that

\[
\tilde\sphericalangle_ka_ipa_j\ge \pi/2-\delta,\,\tilde\sphericalangle_ka_ipb_j\ge \pi/2-\delta,\, \tilde\sphericalangle_kb_ipb_j\ge \pi/2-\delta
\]
for all $i\ne j$,

\[
\tilde\sphericalangle_ka_ipb_i\ge \pi-\delta
\]
and $|pa_i|=|pb_i|=A$ for all $i$. If $p$ is $(m,A,\delta)$-strained and $\delta$ is sufficiently small then $m\le n$.

In this paper we will only consider $(n,A,\delta)$-strained  points. Therefore, to simplify notations we will drop $n$ and refer to such points as $(A,\delta)$-strained. Further, we'll call $p$ $\delta$-strained if it is $(A,\delta)$-strained for some $A>0$. Lastly, we will call $p$ \emph{Euclidean} if $T_pX\cong \R^n$ or, equivalently, if $p$ is $\delta$-strained for any $\delta>0$.

The following Lemma is fundamental

\begin{lem}\label{lem-strainer-map}\cite{BGP}
There are $r_0=r_0(n,k)\ll 1, A_0=A_0(n,k)\gg 1$ such that the following holds.

Suppose $\curv X^n\ge k$ and  $p\in X$ is  $(Ar,\delta)$-strained with $\delta<1/2n$ and  $r\le r_0, A\ge A_0$. Let  $a_1,\ldots, a_n, b_1,\ldots, b_n$ be the corresponding strainer.

Consider the map $F\co X\to\R^n$ given by the formula
\[
F(x)=(|xa_1|,\ldots, |xa_n|)
\]

Then $F$ is $(1\pm c(n) \delta)$-bi-Lipschitz from $B_r(p)$ onto an open neighbourhood of $F(p)\in \R^n$.

\end{lem}










\subsection{DC functions on Alexandrov spaces}
$DC$ functions on Alexandrov spaces are defined in exactly the same way as $DC$ functions on $\R^n$. That is, a function $f$ on an open set $U$ in an Alexandrov space $X$ is $DC$ if  it can be locally represented as  $h-g$ where $h$ and $g$ are locally Lipschitz and semiconcave.
% We call a $DC$ function $DC_0$ if it's continuous outside of  a set $S$ of measure zero.

In~\cite{Per-DC} Perelman proved the following

\begin{thm}\label{thm-DC-Per}
Let $X^n$ be an $n$-dimensional Alexandrov space of $\curv\ge k$. Let $p\in X$ be an $\delta_0$-strained point where $0<\delta_0=\delta_0(n)\ll 1$. Let $ (a_1,b_1,),\ldots, (a_n,b_n)$ be an $(Ar,\delta_0)$ strainer at $p$ such that $\tilde\sphericalangle_ka_ipa_j\ge \pi/2+\delta_0$ for all $i\ne j$ and $A=A(n,k,\delta_0)\gg 1$.

Let  $F=(f_1,\ldots f_m)=(|\cdot a_1|,\ldots, |\cdot a_n|)$ be the strainer coordinates on $B_r(p)$, $r\le 1$.

Then

\begin{enumerate}
\item $f\co B_r(p)\to \R$ is $DC$ if and only if $f\circ F^{-1}$ is $DC$ on $U=F(B_r(p))$.
\item\label{semiconcave} If  $ |pq|\ge 2r$ and  $\tilde\sphericalangle_k qpa_j\ge \pi/2+\delta_0,  j=1,\ldots, n$ then for  $h=|\cdot q|-|pq|,$   it holds that
\[
\bar h=h\circ F^{-1} \text{ is } \lambda=\lambda(n, k,\delta_0,r)-\text{concave on } U.
\]
%\item\label{uniform-DC} If $f=|\cdot q|$ where $|pq|\ge 2r$ then $f\circ F^{-1}$ is \emph{\bf uniformly} $DC$, that is it can be represented as a difference $h-g$ where both $h, g$ are $\lambda=\lambda(k,\delta,A,r)$-concave, $\lambda=\lambda(k,\delta,A,r)$-Lipschitz  on $U$.
 \end{enumerate}
\end{thm}

Using Theorem~\ref{thm-DC-Per} the set $X_{\delta_0}$ of $\delta_0$-strained points in $X$ can be given the structure of a DC manifold. Moreover, Theorem~\ref{thm-DC-Per} remains true if we change  $f_i(x)$ to $\oint_{B_{\eps r}(a_1)}|xy|dvol(y)$ with $\eps\ll 1$ which we will assume from now on.
% and $h$ in part \eqref{semiconcave}   to $h(x)=\oint_{B_{\eps r}(q)}|xy|dvol(y)-|pq|$ where $\eps\ll 1$.

Then the $DC$-coordinate map $F$ becomes $C^1$ on $B_r(p)\backslash S$ where $S$ is the set of singular points in $X$ (i.e. points $x$ where $T_xX\ncong \R^n$).
Then one can talk about $DC^1$-functions on $X$, i.e. $DC$ functions which are continuously differentiable on $X\backslash S$ and $BV_0$- functions, i.e. $BV$ functions which are continuous on $X\backslash S$.
In local coordinates given by $F$ these correspond to $DC^1$ (respectively $BV^0$) functions on $U=F(B_r(p))\subset \R^n$.
Partial derivatives of $DC^1$ functions are $BV^0$.

Any $BV^0$ function $u$ satisfies $\|du\|(A)=0$ for any $A\subset X\setminus S$ which is $\mathcal H_{n-1}$  $\sigma$-finite. If $u$ is $BV^0$ and $v$ is continuous outside a $\mathcal H_{n-1}$  $\sigma$-finite then
$v\cdot\frac{\partial u}{\partial x_i}$ is still a signed Radon measure.

Using $DC^1$ structure, the metric on $X$ induces a $BV^0$ Riemannian metric on $U$ \cite{Per-DC}. This metric can be computed as follows.

Let $h(x)=|xq|$ for some $q\in X$. Then $|\nabla h|=1$ a.e. on $X$ which means that

\begin{equation*}\label{metric-eq-1}
\sum_{ij}g^{ij}\cdot \frac{\partial \bar h}{\partial x_i}\cdot \frac{\partial \bar h}{\partial x_j}= 1 \text{ a.e. on } U
\end{equation*}

It's easy to choose  functions $h_s=|\cdot q_s|-|q_sp|, s=1,\ldots,  \frac{n(n+1)}{2}$, such that the system

\begin{equation}\label{metric-eq-system}
\sum_{ij}g^{ij}\cdot \frac{\partial \bar h_s}{\partial x_i}\cdot \frac{\partial \bar h_s}{\partial x_j}= 1 \quad s=1,\ldots,  \frac{n(n+1)}{2}
\end{equation}
viewed as a linear system of equations for $g^{ij}$ has determinant bounded away from zero on $B_p(r)$. Moreover, we can easily choose the points $q_s$ to satisfy $|pq_s|=Ar$ for all $s$ and   $\tilde\sphericalangle_k q_spa_j\ge \pi/2+\delta_0$ for all $j,s$.
By Theorem~\ref{thm-DC-Per} this implies that all $\bar h_s$ are uniformly semi-concave on $U$.

Solving \eqref{metric-eq-system} for $g^{ij}$ we see that in our $DC^1$ coordinates the metric $g$ can be expressed as

\begin{equation}\label{metric}
g_{ij}=\frac{P_{ij}(\frac{\partial \bar h_s}{\partial x_t})}{Q_{ij}(\frac{\partial \bar h_s}{\partial x_t})}\quad s=1,\ldots, \frac{n(n+1)}{2},t=1,\ldots, n
\end{equation}

where %$\bar h_s$ are of the form as above,that is $h_s(x)=|xq_s|-|pq_s|$, with  $|pq_s|=Ar$, $\bar h_s=h_s\circ F^{-1}$
 $P,Q$ are polynomials of degree $\frac{n(n+1)}{2}$ with uniformly $C(n,\delta_0,k)$-bounded coefficients and the denominator is $C(n,\delta_0,k)$-uniformly bounded away from $0$ on $U$.


In particular, it follows that $g$ is $BV^0$ and
\begin{equation}\label{g'-bounded}
\int_U\|g'\|<\infty
\end{equation}
%(in fact, it's uniformly bounded by $C(n,r,k,A,\delta)$).

Moreover, by construction, if $p$ is $(Ar,\delta)$-strained and $r=1,k\to 0,\delta\to 0$ and $A\to\infty$ then $g_{ij}$ converges to the standard Euclidean metric (more precisely, $\bar h_j$'s converge to the standard Busemann functions on $\R^n$ which produce the standard Euclidean metric when plugged into \eqref{metric}).



The following general Lemma is well-known.
\begin{lem}\label{concave-der-conv}
Let $u_m\co B_2(0)\to\R$, $m=1,2\ldots$ be a sequence of continuous $\lambda$-concave functions converging to $u$.
Then $D^2u_m\to D^2 u$ weakly on $B_1(0)$.
\end{lem}

Applying this lemma to the above situation we obtain the following sharpened version of \eqref{g'-bounded}

 \begin{lem}\label{lem-per-dc-1}
 Let $X^n$ be an $n$-dimensional Alexandrov space of $\curv\ge k$ and let $A>100$.

 If  $p\in X$ is
   $(A,\delta)$-strained, then  in the above $DC_1$-coordinates
the metric tensor $g$ satisfies $\|g'\| (B_{10}(p)) \leq \varkappa( k,\delta,A| n)$ where\\ $\varkappa(k,\delta,A| n)\to 0$ as $k\to 0, \delta\to 0,A\to\infty$.
 %If $A$ goes to $\infty$,  $\delta $
% and the lower curvature bound go to $0$, then $\|g'\| (B_{10} (x))$ goes to $0$, independently of $x$ and $X$.
 \end{lem}



 Let $X^n$ be a compact  $n$-dimensional Alexandrov space with $\curv\ge\kappa$ without boundary. For simplicity we will assume that  $\kappa=0$ and $X$ has $\curv\ge 0$.
 Then absolute volume comparison implies that for any $p\in X$ and $\eps>0$ we have $\vol B_\eps(p)\le \omega_n\eps^n$ and hence $v_\eps\ge 0$ for any $\eps>0$. Thus to prove local finiteness of mm-boundary of $X$ it's sufficient to prove local linear in $\eps$ bound on $v_\eps$ from above.

 Fix a sufficiently small $0<\delta<\delta_0$.
%say, $\delta=1/10^n$ will do.


 %Let $X_{\delta}$ be the set of $\delta$-strained
 %points and $S_{\delta}$ be its complement.
%Let $p\in X_{\delta}$ and let $\{(a_i,b_i)\}_{i=1}^n$ be an $(A,\delta)$ strainer near $p$. Then the distance coordinate map
  %$x\mapsto (|x,a_1|,\ldots |x,a_n|)$ is locally $1+10\delta)$-biLipschitz near $p$ ~\cite{BGP} and according to ~\cite{Per-DC} the collection of such coordinate charts over all $p\in X_{\delta}$ gives $X_{\delta}$ a structure of a $DC^0$ manifold. Moreover, the averaged coordinate charts
  %$$x\mapsto (\oint_{B_\eps(a_1)}|xy|dvol(y),\ldots, \oint_{B_\eps(a_n)}|xy|dvol(y))$$ where $\eps\ll A$ gives $( X_{\delta},R)$ the structure of a $DC^1$-manifold for some subset of full measure $R\subset X_{\delta}$.

\subsection{Bounds on $\mu$}

 Let
 $\mu$ be the measure defined in  section~\ref{sec-BV-estimate}
 (the minimal derivative of the metric).





Corollary~\ref{cor-dc-vr} implies

 \begin{prop} \label{mainalex}
 %The measure $\mu$ defined in the last section satisfies $\mu (X_{\delta } )< \infty$.
 There is a constant $C$,
 such that  for any relatively compact subset $A\subset X_\delta$ for all small
 $0<r\leq r_0 (A)$  it holds that  $v_r (A) \leq C \cdot r\cdot \mu  ( B_{2r} (A))$.
 \end{prop}

Our next goal will be to estimate $\mu$.

% We start with a couple of useful observations. The first lemma is due to Perelman:

% \begin{lem}\label{lem-per-dc}
 %There is a constant $A\gg 1$, such that if $\curv X^n\ge k$, a point $x\in X$  is $(A,\delta)$-strained, then  in the distance
 %coordinates the metric tensor $g$ satisfies $\|g'\| (B_{10}(x)) \leq \kappa(k,\delta,A|n)$ where $\varkappa(k,\delta,A|n)\to 0$ as $k\to 0, \delta\to 0,A\to\infty$.
 %If $A$ goes to $\infty$,  $\delta $
% and the lower curvature bound go to $0$, then $\|g'\| (B_{10} (x))$ goes to $0$, independently of $x$ and $X$.
 %\end{lem}

By rescaling Lemma~\ref{lem-per-dc-1} gives

 \begin{cor}  \label{ballmeasure}
 If a point $x\in X$ is $(Ar,\delta )$-strained, then $\|g'\| (B_{10r} (x) ) \leq \varkappa(k,\delta,A)  \cdot r^{n-1}$ where $\varkappa(k,\delta,A|n)\to 0$ as $k\to 0, \delta\to 0,A\to\infty$.
 \end{cor}



As the first immediate consequence we get:

\begin{cor}\label{cor-n-1-density}
The measure $\mu$ has finite $(n-1)$-dimensional density. It vanishes on all subsets of $X_\delta$ of finite $(n-1)$-dimensional
Hausdorff measure.
\end{cor}

\begin{proof}
The locally finite measure $\mu$ has a finite $(n-1)$-dimensional density due to the preceding corollary.
The set of non-Euclidean points has Hausdroff dimension at most $(n-2)$, thus its $\mu$-measure is $0$.
At any Euclidean point, the $(n-1)$-dimensional density is $0$, due to Corollary~\ref{ballmeasure}.
\end{proof}


\subsection{Good and bad balls}
Let $A$ be the constant supplied by Lemma ~\ref{lem-per-dc}.

A ball $B(x,R)$ in $X^n$ will be called \emph{good}
if any point $p\in B(x,2 R)$ is  $(AR,\delta)$-strained with respect to a fixed
collection of points say $a_1,\dots a_n,b_1,\dots,b_n$.
Recall that in this case
the distance map $B(x,r)\to \RR^m$
$$p\mapsto(|a_1p|,\dots,|a_np|)$$
is bi-Lipschitz with bi-Lipschitz constant close to 1.

A ball $B(x,r)$ in $X$ will be called \emph{bad} if it is not good.

Let us denote by $\alpha_m$ the volume of unit ball in the $m$-dimensional Euclidean space.

Combining Corollaries ~\ref{cor-mu-dc} and ~\ref{ballmeasure} we obtain:
\begin{lem}\label{lem:good-ball}

Given $n$ and $\kappa$ there exist
 $C=C(n,\kappa)$ and $R_0=R_0(n,\kappa)$ such that if $X^n$ is an $n$-dimensional Alexandrov space of $\curv\ge\kappa$% without a boundary

 then for any good ball $B(x,R)\subset X$ with $R\le R_0$ and any $W\subset B(x,R)$  we have
 \[
 v_r(W)\le Cr R^{n-1}
 \]


for any $r<R$.
\end{lem}



\begin{prop}\label{prop:covering}
Let $X^n$ be an $n$-dimensional Alexandrov space.
Then any bounded set $W\subset X$ of strained points can be covered by at most countable collection of good balls $B_n=B(x_m,r_m)$
such that
$$\sum_m r_m^{n-\frac32}<\infty.$$

\end{prop}

The proof is obtained by recursive applictaion of the following Lemma.

\begin{lem}\label{lem:covering}
Let $X$ be an $m$-dimensional Alexandrov space without boundary,
$x_0\in X$
and $r_0>0$.
Then there is an integer $N$ such that for any point $x\in B(x_0,r_0)$ and radius $r<r_0$
the ball $B(x,r)$ can be covered by at most $N$ balls
balls $B_i=B(x_i,r_i)$ such that
$$\sum_{i\in \BAD}r_i^{n-\frac32}< \tfrac12\cdot r^{n-\frac32}.$$
where $i\in \BAD$ means that $B_i$ is a bad ball.

\end{lem}

\parit{Proof.}
Assume contrary;
i.e., there is a sequence of the balls
$B_n=B(p_m,\rho_m)$
such that $p_m\in B(x_0,r_0)$,
$\rho_m<r_0$ and
one needs at least $m$ balls to cover $B_m$ so that the conditions in the lemma meet.

Pass to a converging subsequece in the pointed Hausdorff metric.
$$(\tfrac1{\rho_m}\cdot X,p_m)\GHto (X_\infty,p).$$
From Bishop--Gromov inequality we have a lower bound on the voulume of unit ball in $X_\infty$ centered at $p$.
In particular the sequence $(\tfrac1{\rho_m}\cdot X,p_m)$
is not collapsing.
It follows that $X_\infty$ is an $m$-dimensional Alexandrov space
and by Perelman's stability theorem, $X_\infty$ has no boundary.

By \cite{BGP}, the set of $\delta$-singular points $S$ in $X$
is a closed set with Hausdorff dimension $\le n-2$.
In particular, we can cover $S\cap \bar B[p,1]$ by a finite number of balls
$B_i=B(x_i,r_i)$ such that the sum
$$\sum_ir_i^{n-\frac32}$$
is arbitrary small.
In particular we can assume
$$\sum_ir^{n-\frac32}<(\tfrac12)^{n-\frac32}.$$

Any point in the remaining set $R=\bar B[p,1]\backslash \cup_i B_i$
is $\delta$-regular.
Therefore a small ball centered at any point in $R$ is good.
Since $R$ is compact,
we can cover it by finite number of good balls.
Let $N$ be the total number of balls in the obtained covering of $\bar B[p,1]$

Lifting the constructed covering to $B_m$,
we get that for any large $m$
the ball $B_m$ can be covered by at most $N$ balls satisfying the lemma,
a contradiction.
\qeds

\section{Proof of Theorem~\ref{alexandrovthm}}

\parit{Proof.}

Let us first consider the case when $X^n$ is a compact Alexandrov space   without boundary.

Let $B_m=B(x_m,r_m)$ be the collection of balls as in Proposition~\ref{prop:covering}.

Applying Lemma~\ref{lem:good-ball},
we get
$$v_r(B_m)\le Cr\cdot r_m^{n-1}$$
if $r<r_m$.

On the other hand,
\begin{align*}
\sum_{\set{m}{r_m<r}}r_m^n
&\le r^{\frac32}\cdot\sum r_m^{n-\frac32}\le
\\
&\le C\cdot r^{\frac32}.
\end{align*}

Denote by $W'$ the set $W$ with all the singular points removed.
Since $\vol_n(W\backslash W')=0$ we get
\[v_r(W')=v_r(W)\]
for any $r>0$.

Let us subdivide $W'$ into subsets $W_m'$ so that
$W_m'\subset B_m$ for each $m$.
From above we get the following.
\begin{align*}
v_r(W)
&= \sum_m v_r W_m'=
\\
&\le \sum_{r_m\ge r} v_r W_m'
+
 \sum_{r_m< r} v_r W_m'\le  \sum_{r_m\ge r}Crr_m^{n-1}+Cr^{\frac 3 2}
\\
&\le Cr
\end{align*}
The last inequality follows by  Proposition~\ref{prop:covering}.


%since  $$\sum_{r_n< r} \vol_m W'_n\le \alpha_m\cdot \sum_{r_n< r}r_n^m.$$

The case of when $\partial X\ne\emptyset$ follows from above by using that the $r$-neighborhood of $\partial X$ has $\vol_n\le r\cdot \vol_{n-1}\partial X$.

Part \eqref{full-measure-zero-nu} of Theorem~\ref{alexandrovthm}  follows from the fact (see ~\cite{Per-DC}) that at any point $p\in X$ at which $g$ is differentiable in appropriate coordinates $G$ it holds that

\[
g_{ij}(G(x))=\delta_{ij}+o(|px|)
\]

and

\[
 \big| |xy|-|G(x)G(y)|\big|=o(r^2) \text{ for all } x,y\in B_r(p)
\]
This easily implies that the densitty of $\nu$ at $p$ with respect to $\mathcal H^n$  is zero.

Part \eqref{n-1-nu} follows directly from Corollary~\ref {cor-n-1-density}. To see part \eqref{bry-nu}  observe that  for every regular point $p\in\partial X$ there exist an $n$-strainer map $f$ on a neighborhood $U=B_R(p)$ of $p$ in $X$ of the form $f(x)=(f_1(x),\ldots, f_n(x))=(|x\partial X, |xa_1|,\ldots |x a_{n-1}|)$ which is $(1\pm\delta)$-bi-Lipschitz on $U$ and maps $U$ onto an open subset in the half-space in $\R^n$ preserving the boundary.

Using these coordinates it's easy to see that we have the following.

\[
\vol_n(f_1\le r)=(1\pm\delta)r\vol_{n-1}(U\cap\partial X)
\]

and for any $x\in \{f_1\le r/10\}$ we have that $\vol_n(B_r(x))=\frac{\omega_nr^n(1\pm\delta)}{2}$.

Putting the above inequalities together we immediately conclude that $v_r( \{f_1\le r/10\})\ge rc_n\vol_{n-1}(U\cap\partial X)$ which immediately implies \eqref{bry-nu} .

\qedsf

\section{Questions and Comments} \label{sec:final}
We start with some open questions, which hopefully will be addressed elsewhere.










\begin{quest} Let $X$ be the boundary of a convex body in $\R ^n$ with the induced inner metric.
Does it have finite mm-curvature?  Can it maybe be bounded by a function of its total scalar curvature
in the sense of integral geometry?
\end{quest}


\begin{quest}
Can one express the mm-curvature of an Alexandrov  surface in terms of its curvature measure?
\end{quest}

\begin{quest}
Does the mm-boundary vanish  in smoothable Alexandrov spaces?   Are there relations to scalar curvature measures defined in \cite{LP}?
\end{quest}


\begin{quest}
Can one estimate and use the mm-boundary in geodesically complete spaces with upper curvature bounds to study the geodesic flow?
\end{quest}



\begin{quest}
 Can one construct a compact Riemannian manifold with a continuous Riemannian metric
that does not have finite mm-boundary?
\end{quest}
{\color{red}
\begin{quest}
Is the $mm$-measure zero (finite) for noncollapsed limits of Riemannian manifolds with Ricci curvature bounded below? More generally, is it zero (finite) for appropriately defined noncollpased $RCD(k,N)$ spaces without boundary?
\end{quest}
}

\bibliographystyle{alpha}
\bibliography{mmm}


\end{document}



An important geometric ingredient is the following quantitative estimate of the set of singular points:

\begin{lem} \label{count}
 There is a number $C>0$ such that any $r$-net in the space $Y_{Ar,\delta}$ of not $(Ar,\delta)$-strained points
 has at most $C r^{2-n}$ elements.
\end{lem}


 Using this Lemma and  \cref{ballmeasure} one obtains the proof of \pref{mainalex} by subdividing
 $X$ as $X=X_r \cap (X_r \setminus X_{\frac r 2} ) \cup .....$.









